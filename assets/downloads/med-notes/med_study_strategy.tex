\documentclass{article}
\usepackage[margin=0.5in]{geometry}
\usepackage{amsmath, graphicx, multicol, amsfonts, amsthm, hyperref, booktabs, hyperref}
\usepackage{caption} 
\captionsetup[table]{skip=5pt}

\frenchspacing
\setlength{\parindent}{0pt}

\begin{document}

\title{My Study Strategy for Medical School at UCalgary}
\author{Eddie Guo}
\date{\today}

\maketitle

\section{Resources I Use}

\textit{Conflict of interest declaration:} I use these resources because I find them helpful, not because I am paid to do so. I am not affiliated with any of these resources. I am collaborating with Toronto Notes editors to create AI-generated practice questions.

\begin{table}[h]
    \centering
    \caption{Resources I use in order of frequency (from top to bottom).}
    \begin{tabular}{llll}
        \toprule
        \textbf{QBanks} & \textbf{Videos} & \textbf{Books} & \textbf{Apps} \\
        \midrule
        \begin{tabular}{p{1.5cm}}
            Ankihub \\
            UWorld \\
            AMBOSS
        \end{tabular} &
        \begin{tabular}{p{1.25cm}}
            Sketchy \\
            Osmosis
        \end{tabular} &
        \begin{tabular}{p{5.25cm}}
            First Aid for the USMLE Step 1 \\
            First Aid for the USMLE Step 2 \\
            Netter's Clinical Anatomy \\
            Toronto Notes
        \end{tabular} &
        \begin{tabular}{p{3cm}}
            Anki \\
            \href{https://www.perplexity.ai/}{perplexity.ai}$^*$ \\
            Complete Anatomy
        \end{tabular} \\
        \bottomrule
        \multicolumn{4}{l}{\small $^*$perplexity.ai is a GPT-powered tool that provides sources with its answers.}
    \end{tabular}
\end{table} \vspace{-1em}

\section{My Study Strategy}

\subsection{Learning New Concepts}

\begin{enumerate}
    \item Sketchy: great for memorizing lots of things at once using method of loci (memory palace)
    \begin{itemize}
        \item If Sketchy video not available, I watch the UME podcast
    \end{itemize}
    \item Do the practice Sketchy questions ($\sim$10 questions per topic)
    \item Review lecture slides for anything missing from Sketchy
    \item Anki
    \begin{itemize}
        \item Unsuspend relevant Anki cards from the AnKing deck (contains cards for USMLE Step 1 and 2 content)
        \item Create missing cards from lecture slides
        \item \textit{Note:} I do $\sim$120 new cards/day, $\sim$1000 total cards/day (takes just under 2 hours)
    \end{itemize}
    \item Read relevant sections in First Aid
\end{enumerate}

\subsection{Consolidation}

\textbf{ANKI!!!}, then I review Toronto Notes for a quick summary followed by practice questions:
\begin{itemize}
    \item UWorld and AMBOSS have tons of practice questions; AMBOSS has an Anki plugin that gives you cards based on your learned Anki cards. Toronto Notes have fantastic summaries of diseases relevant to UCalgary and MCCQE exams.
    \item Any time there is something I don’t understand, I ask \href{https://www.perplexity.ai/}{perplexity.ai} a question
\end{itemize}

\subsection{Anatomy}

\begin{itemize}
    \item Osmosis videos and UME anatomy note package
    \item Add Anki cards based on Osmosis and UME anatomy note package; add relevant images from Netter's Anatomy
    \item Review 3D anatomy with Complete Anatomy
\end{itemize}

\subsection{Exam Prep}

\begin{itemize}
    \item Brain dump onto paper comparing against MANIC notes, then review anything I missed
\end{itemize}

\end{document}
