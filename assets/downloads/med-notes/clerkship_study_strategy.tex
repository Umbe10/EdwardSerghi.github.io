\documentclass{article}
\usepackage[margin=0.5in]{geometry}
\usepackage{amsmath, graphicx, multicol, amsfonts, amsthm, hyperref, booktabs, hyperref}
\usepackage{caption} 
\captionsetup[table]{skip=5pt}

\frenchspacing
\setlength{\parindent}{0pt}

\begin{document}

\title{My Study Strategy for Clerkship}
\author{Eddie Guo}
\date{\today}

\maketitle

\section*{General Resources}
These are resources that I found useful for all clerkship rotations.
\begin{enumerate}
    \item \textit{UpToDate and DynaMed:} For quick reference, especially handy for clinical algorithms and medication dosing. Also great for deeper dives into epidemiology, pathophysiology, presentation, management, and relevant clinical trials.
    \item \textit{Toronto Notes:} A compact review of all major topics in med school. It's a great review resource for clerkship exams.
    \item \textit{MDCalc:} Super useful for clinical decision tools and calculators (e.g., CHA2DS2-VAsc, 4Ts).
    \item \textit{MD on Call:} An iOS app that provides a quick reference for common things you might be called for (e.g., tachycardia, hyponatremia). It's a great resource for on-the-fly reference. It also has common questions to ask when you are called about a patient.
    \item \textit{Firstline:} Provides location-specific antibiograms and guidelines for common infections. It's a great resource for antibiotic selection.
    \item \textit{Touch Surgery:} A great app for learning common surgical procedures. It provides step-by-step instructions and 3D models of the procedures.
\end{enumerate}

I also found it helpful to read clinical guidelines for common conditions (e.g., heart failure, osteoporosis, stroke). For specialties I was interested in, I would search up and read landmark clinical trials, condition-specific papers (e.g., review paper on WHO brain tumour classification, approach to elevated intracranial pressure), and read relevant sections in specialty-specific textbooks. \vspace{1em}

Specific Anki decks I used:
\begin{multicols}{2}
\begin{enumerate}
    \item \textbf{General decks:}
    \begin{enumerate}
        \item AnKing: Comprehensive deck for Step 1 and Step 2CK
        \item Canki: Canadian Anki deck for MCCQE Part 1
        \item Mehlman Medical
        \item Mehlman HY Arrows - VerifiedSmoothBrain
        \item The DIP Deck (Divine Intervention Podcast)
    \end{enumerate}
    \item \textbf{Specialty decks:}
    \begin{enumerate}
        \item HoggieMed USPSTF
        \item Brand Names
        \item Surgical Instruments
        \item AnKore for ABR Diagnostic Radiology Core Exam
        \item Surgical Recall
        \item Dura Deck v2
        \item Jeff Deck: Neurosurgery Sub-I
        \item Thomas Jefferson AANS Neurosurgery Deck
    \end{enumerate}
    \item \textbf{Anatomy:}
    \begin{enumerate}
        \item 100 Concepts Anatomy
        \item Netter Better
        \item University of Michigan - BlueLink Atlas
    \end{enumerate}
    \item \textbf{Internal Medicine:}
    \begin{enumerate}
        \item Critical Care Medicine
        \item Atlas of Auscultation
        \item The Only EKG Deck You'll Ever Need
    \end{enumerate}
\end{enumerate}
\end{multicols}

\section*{Exam Strategy}
My approach was pretty simple: Anki to retain material, UWorld Step 2CK QBank for practice, and Toronto Notes for review. Throughout the rotation, I would read around patient cases on UpToDate, textbooks, and guidelines. I also made a note of common site-specific management protocols (e.g., blood pressure parameters for certain conditions).

\end{document}
