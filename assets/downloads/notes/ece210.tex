\documentclass[twocolumn]{article}
\usepackage[margin=0.75cm]{geometry}

\usepackage{graphicx, multicol, wrapfig, caption, multirow, mathtools}
\setlength{\columnseprule}{.75pt}
\def\columnseprulecolor{\color{black}}

\usepackage[table]{xcolor}
\usepackage{tabularx, tcolorbox}
\usepackage{karnaugh-map}

% circuits
\usepackage{tikz, circuitikz}
\usetikzlibrary{calc, positioning}

\setlength{\parindent}{0pt}
\setlength{\parskip}{6pt}
\setlength\extrarowheight{2pt}

\everymath{\displaystyle}
\newcommand{\overbar}[1]{\mkern 1.5mu\overline{\mkern-1.5mu#1\mkern-1.5mu}\mkern 1.5mu}

\usepackage{booktabs}

\title{
	\vspace{-2em}
	\normalsize \textbf{ECE 210 Formula Sheet} \\
	\small Eddie Guo \\
	\dotfill
	\vspace{-5em}
}
\date{}

\begin{document}
\maketitle
\rowcolors{1}{gray!10}{white}

\textbf{Number Systems}

$N_{10} \to N_R$: Do $N/R$ until 0 then build back using remainder.

$(0 < N_{10} < 1) \to N_R$: Do $N \times R$ until number of terms sufficient and build sequentially using the ones digit.

1's complement: add last carry to right \hfill $\overbar{N} = (2^n-1)-N$

2's complement: flip bits then add 1 \hfill $N^* = 2^n - N$

\vspace{-.5em}

\dotfill

\textbf{Boolean Algebra}

\begin{tcolorbox}[width=\columnwidth, sharp corners, tabularx={*{2}{X}l}]
    \cellcolor{black} \textbf{\textcolor{white}{Identities}} & \cellcolor{black} \\
    Identities & $X+0=X$ \\
    & $X \cdot 0 = 0$ \\
    & $X + 1 = 1$ \\
    & $X \cdot 1 = X$ \\
    Idempotency & $X+X=X$ \\
    & $X \cdot X = X$ \\
\end{tcolorbox}

\vspace{.5em}

\begin{tcolorbox}[width=\columnwidth, sharp corners, tabularx={*{2}{X}l}]
    \cellcolor{black} \textbf{\textcolor{white}{Negation}} & \cellcolor{black} \\
    Complementarity & $X + X' = 1$ \\
    & $X \cdot X' = 0$ \\
    Involution & $(X')' = X$ \\
\end{tcolorbox}

\vspace{.5em}

\begin{tcolorbox}[width=\columnwidth, sharp corners, tabularx={*{2}{X}l}]
    \cellcolor{black} \textbf{\textcolor{white}{Laws}} & \cellcolor{black} \\
    Commutativity & $A \cdot B = B \cdot A$ \\
    & $A + B = B + A$ \\
    Associativity & $A \cdot (B \cdot C) = (A \cdot B) \cdot C$ \\
    & $A + (B + C) = (A + B) + C$ \\
    Distributivity & $A \cdot (B + C) = A \cdot B + A \cdot C$ \\
    & $A + B \cdot C = (A + B) (A + C)$ \\
\end{tcolorbox}

\vspace{.5em}

\begin{tcolorbox}[width=\columnwidth, sharp corners, tabularx={*{2}{X}l}]
    \cellcolor{black} \textbf{\textcolor{white}{De Morgan's Laws}} & \cellcolor{black} \\
    \toprule
    $(X \cdot Y)' = X' + Y'$ & $\left( \prod_{i=1}^n X_i \right)' = \sum_{i=1}^n X_i'$ \\
    $(X + Y)' = X' \cdot Y'$ & $\left( \sum_{i=1}^n X_i \right)' = \prod_{i=1}^n X_i'$ \\
\end{tcolorbox}

\vspace{.5em}

\begin{tcolorbox}[width=\columnwidth, sharp corners, tabularx={*{1}{X}l*{1}{X}l}]
    \cellcolor{black} \textbf{\textcolor{white}{Theorems}} & \cellcolor{black} \\
    Uniting & $XY + XY' = X$ \\
    & $(X+Y)(X+Y') = X$ \\
    Absorption & $X + XY = X$ \\
    & $X(X+Y) = X$ \\
    Elimination & $X + X'Y = X + Y$ \\
    & $X(X' + Y) = XY$ \\
    Consensus & $XY + X'Z + YZ = XY + X'Z$ \\
    & $(X+Y)(X'+Z)(Y+Z) = (X+Y)(X'+Z)$ \\
    Other & $(X+Y)(X'+Z) = XZ + X'Y$ \\
\end{tcolorbox}

\newpage

\textbf{Binary Codes}

Be careful about multiple ways to form a number!

\begin{tcolorbox}[width=\columnwidth, sharp corners, tabularx={*{8}{X}l}]
    \cellcolor{black} \textbf{\textcolor{white}{Dec}} & \cellcolor{black} \textbf{\textcolor{white}{BCD}} & \cellcolor{black} \textbf{\textcolor{white}{6311}} & \cellcolor{black} \textbf{\textcolor{white}{XS-3}} & \cellcolor{black} \textbf{\textcolor{white}{2/5}} & \cellcolor{black} \textbf{\textcolor{white}{Gray}} \\
    0 & 0000 & 0000 & 0011 & 00011 & 0000 \\
    1 & 0001 & 0001 & 0100 & 00101 & 0001 \\
    2 & 0010 & 0011 & 0101 & 00110 & 0011 \\
    3 & 0011 & 0100 & 0110 & 01001 & 0010 \\
    4 & 0100 & 0101 & 0111 & 01010 & 0110 \\
    5 & 0101 & 0111 & 1000 & 01100 & 0111 \\
    6 & 0110 & 1000 & 1001 & 10001 & 0101 \\
    7 & 0111 & 1001 & 1010 & 10010 & 0100 \\
    8 & 1000 & 1011 & 1011 & 10100 & 1100 \\
    9 & 1001 & 1100 & 1100 & 11000 & 1101 \\
\end{tcolorbox}

\vspace{-.5em}

\dotfill

\textbf{Look Up Tables}

$n$ inputs \hfill $\Longrightarrow$ \hfill $2^n$ entries/states \hfill $\Longrightarrow$ \hfill $2^{2^n}$ possible fns

\textbf{Caution:} Write out the formula (e.g., $2^{2^n}$) for full marks

If $n=6$, how many fns take 2 or fewer inputs? \vspace{-.5em}
\begin{itemize}
    \item Exactly 0 inputs: $2^{2^0} = 2$ fns
    \item Exactly 1 input: $2^{2^1} - 2^{2^0} = 2$ fns
    \subitem Choose 1 of 6 inputs: $N = {6 \choose 1} \times 2 = 6 \times 2 = 12$
    \item Exactly 2 inputs: $2^{2^2} - 2^{2^1} - 2^{2^0} = 10$ fns
    \subitem Choose 2 of 6 inputs: $N = {6 \choose 2} \times 2 = 15 \times 10 = 150$
    \item $N = 2 + 12 + 150 = 164$ fns total
\end{itemize} \vspace{-1em}

\vspace{-.5em}

\dotfill

\textbf{SOP and POS}

SOP: multiply out \hfill if one term is 1, whole expr is 1

POS: factor \hfill if one term is 0, whole expr is 0

\vspace{-.5em}

\dotfill

\textbf{Logic Gates}

\begin{circuitikz}
    % gates
    \node[ieeestd not port, number inputs=1, scale=0.5] (A) at (0, 0) {};
    \node[ieeestd and port, number inputs=2, scale=0.5] (B) at (0, -1) {};
    \node[ieeestd or port, number inputs=2, scale=0.5] (C) at (0, -2) {};
    \node[ieeestd nand port, number inputs=2, scale=0.5] (D) at (0, -3) {};
    \node[ieeestd nor port, number inputs=2, scale=0.5] (E) at (0, -4) {};
    \node[ieeestd xor port, number inputs=2, scale=0.5] (F) at (0, -5) {};
    \node[ieeestd xnor port, number inputs=2, scale=0.5] (G) at (0, -6) {};

    % NOT labels
    \node[left] at (A.in 1) {$x$};
    \node[right] at (A.out) {$x'$};

    % AND labels
    \node[left] at (B.in 1) {$x$};
    \node[left] at (B.in 2) {$y$};
    \node[right] at (B.out) {$x y$};

    % OR labels
    \node[left] at (C.in 1) {$x$};
    \node[left] at (C.in 2) {$y$};
    \node[right] at (C.out) {$x + y$};

    % NAND labels
    \node[left] at (D.in 1) {$x$};
    \node[left] at (D.in 2) {$y$};
    \node[right] at (D.out) {$(xy)' = x' + y'$};

    % NOR labels
    \node[left] at (E.in 1) {$x$};
    \node[left] at (E.in 2) {$y$};
    \node[right] at (E.out) {$(x+y)' = x'y'$};

    % XOR labels
    \node[left] at (F.in 1) {$x$};
    \node[left] at (F.in 2) {$y$};
    \node[right] at (F.out) {$x \oplus y = x'y + xy'$};

    % EQUIV labels
    \node[left] at (G.in 1) {$x$};
    \node[left] at (G.in 2) {$y$};
    \node[right] at (G.out) {$x \otimes y = (x \equiv y) = (x \oplus y)' = xy + x'y'$};

    % gate labels
    \node[left] at ($(A.out) + (-2.35, 0)$) {NOT};
    \node[left] at ($(B.out) + (-2.5, 0)$) {AND};
    \node[left] at ($(C.out) + (-2.5, 0)$) {OR};
    \node[left] at ($(D.out) + (-2.5, 0)$) {NAND};
    \node[left] at ($(E.out) + (-2.5, 0)$) {NOR};
    \node[left] at ($(F.out) + (-2.5, 0)$) {XOR};
    \node[left] at ($(G.out) + (-2.5, 0)$) {XNOR};
\end{circuitikz}


\cleardoublepage


\textbf{Logic Gate Heuristics}

NAND is inverted OR \hfill $f = (xy)' = x' + y'$

NOR is inverted AND \hfill $f = (x+y)' = x'y'$

NAND returns 0 $\iff$ all inputs 1

NOR returns 1 $\iff$ all inputs 0

XOR returns 1 $\iff$ inputs have odd number of 1s

XNOR returns 1 $\iff$ all inputs equivalent

\begin{tcolorbox}[width=\columnwidth, sharp corners, tabularx={*{1}{X}l}]
    \cellcolor{black} \textbf{\textcolor{white}{XOR Properties}} \\
    $X \oplus 0 = X$ \hfill $X \oplus 1 = X'$ \\
    $X \oplus X = 0$ \hfill $X \oplus X' = 1$ \\
    $X \oplus Y = Y \oplus X$ \\
    $(X \oplus Y) \oplus Z = X \oplus (Y \oplus Z)$ \\
    $X(Y \oplus Z) = XY \oplus XZ$ \\
    $(X \oplus Y)' = X \oplus Y' = X' \oplus Y = XY + X'Y' = (X \equiv Y)$
\end{tcolorbox}

\begin{tcolorbox}[width=\columnwidth, sharp corners, tabularx={*{1}{X}l}]
    \cellcolor{black} \textbf{\textcolor{white}{XNOR Properties}} \\
    $X \otimes 0 = X'$ \hfill $X \otimes 1 = X$ \\
    $X \otimes X = 1$ \hfill $X \otimes X' = 0$ \\
    $X \otimes X \otimes \cdots \otimes X = \begin{dcases} X, & n \text{ odd} \\ 1, & n \text{ even} \end{dcases}$
\end{tcolorbox}

XOR and XNOR complements if even no of inputs given

$\quad A \oplus B \oplus C \oplus D = \overbar{A \otimes B \otimes C \otimes D}$

XOR and XOR same if odd no of inputs given

$\quad A \oplus B \oplus C = A \otimes B \otimes C$

\vspace{-.5em}
\dotfill

\textbf{Minterms and Maxterms}

\begin{tcolorbox}[width=\columnwidth, sharp corners, tabularx={*{4}{X}l}]
    \cellcolor{black} \textbf{\textcolor{white}{Row}} & \cellcolor{black} \textbf{\textcolor{white}{A B C}} & \cellcolor{black} \textbf{\textcolor{white}{Minterms}} & \cellcolor{black} \textbf{\textcolor{white}{Maxterms}} \\
    0 & 0 0 0 & $A'B'C'$ & $A + B + C$ \\
    1 & 0 0 1 & $A' B' C$ & $A + B + C'$ \\
    $\vdots$ & $\vdots$ & $\vdots$ & $\vdots$ \\
    9 & 1 1 1 & $A B C$ & $A' + B' + C'$ \\
\end{tcolorbox}

Q: Find minterms and maxterms of $f(a, b, c, d) = a'b'$.

A: Since 4 inputs, $2^4=16$ possible outputs.
\begin{align*}
    \text{minterms} &= a'b' \underbrace{(c+c') (d+d')}_{\text{equals 1}} \\
    &= \underbrace{a'b'cd}_{0011} + \underbrace{a'b'cd'}_{0010} + \underbrace{a'b'c'd}_{0001} + \underbrace{a'b'c'd'}_{0000} \\
    &= \sum m(0, 1, 2, 3) \\
    \text{maxterms} &= \prod M(4, 5, \dots, 15)
\end{align*}

If $f = \sum m(3, 4, 5, 6, 7)$, \vspace{-.5em}
\begin{itemize}
    \item Maxterm expansion of $f$ is $\prod M(0, 1, 2)$.
    \item Minterm expansion of $f'$ is $\sum m(0, 1, 2)$.
    \item Maxterm expansion of $f'$ is $\prod M(3, 4, 5, 6, 7)$.
\end{itemize} \vspace{-.5em}


\newpage

\textbf{NAND as a Universal Gate}

\begin{circuitikz}
    % gates
    \node (text) at (-3, 0) {NAND as NOT};
    \node[ieeestd nand port, number inputs=2, scale=0.5] (A) at (0, 0) {};

    % nand inputs
    \draw (A.in 1) -- (A.in 2);
    \draw (A.in 1 |- A.out) to (-.75, 0);
\end{circuitikz}

\begin{circuitikz}
    % gates
    \node (text) at (-4.5, 0) {NAND as AND};
    \node[ieeestd nand port, number inputs=2, scale=0.5] (A) at (-1.5, 0) {};
    \node[ieeestd nand port, number inputs=2, scale=0.5] (B) at (0, 0) {};

    % nand inputs
    \draw (B.in 1) -- (B.in 2);
    \draw (B.in 1 |- B.out) to (-.75, 0) to (A.out);

    % nand inputs
    \node[left] at (A.in 1) {};
    \node[left] at (A.in 2) {};
\end{circuitikz}

\begin{circuitikz}
    % gates
    \node (text) at (-3.125, 0) {NAND as OR};
    \node[ieeestd nand port, number inputs=2, scale=0.5] (A) at (0, .35) {};
    \node[ieeestd nand port, number inputs=2, scale=0.5] (B) at (0, -.35) {};
    \node[ieeestd nand port, number inputs=2, scale=0.5] (C) at (1.5, 0) {};

    % nand inputs
    \draw (A.in 1) -- (A.in 2);
    \draw (A.in 1 |- A.out) to (-.75, .35);
    \draw (B.in 1) -- (B.in 2);
    \draw (B.in 1 |- B.out) to (-.75, -.35);
    \draw (A.out) -| (C.in 1);
    \draw (B.out) -| (C.in 2);
\end{circuitikz}

\begin{circuitikz}
    % gates
    \node (text) at (-3, 0) {NAND as NOR};
    \node[ieeestd nand port, number inputs=2, scale=0.5] (A) at (0, .35) {};
    \node[ieeestd nand port, number inputs=2, scale=0.5] (B) at (0, -.35) {};
    \node[ieeestd nand port, number inputs=2, scale=0.5] (C) at (1.5, 0) {};
    \node[ieeestd nand port, number inputs=2, scale=0.5] (E) at (3, 0) {};

    % nand inputs
    \draw (A.in 1) -- (A.in 2);
    \draw (A.in 1 |- A.out) to (-.75, .35);
    \draw (B.in 1) -- (B.in 2);
    \draw (B.in 1 |- B.out) to (-.75, -.35);
    \draw (A.out) -| (C.in 1);
    \draw (B.out) -| (C.in 2);
    \draw (E.in 1) -- (E.in 2);
    \draw (E.in 1 |- E.out) to (C.out);
\end{circuitikz}

\begin{figure}[h]
    NAND as XNOR \hfill \includegraphics[width=0.65\columnwidth]{Figures/xnor_from_nand.png}
\end{figure}

\begin{figure}[h]
    NAND as XOR \hspace{2em} \includegraphics[width=0.5\columnwidth]{Figures/xor_from_nand.png}    
\end{figure}


\vspace{-.5em}
\dotfill

\textbf{NOR as a Universal Gate}

\begin{circuitikz}
    % gates
    \node (text) at (-3.215, 0) {NOR as NOT};
    \node[ieeestd nor port, number inputs=2, scale=0.5] (A) at (0, 0) {};

    % nor inputs
    \draw (A.in 1) -- (A.in 2);
    \draw (A.in 1 |- A.out) to (-.75, 0);
\end{circuitikz}

\begin{circuitikz}
    % gates
    \node (text) at (-3.125, 0) {NOR as AND};
    \node[ieeestd nor port, number inputs=2, scale=0.5] (A) at (0, .35) {};
    \node[ieeestd nor port, number inputs=2, scale=0.5] (B) at (0, -.35) {};
    \node[ieeestd nor port, number inputs=2, scale=0.5] (C) at (1.5, 0) {};
    
    % nor inputs
    \draw (A.in 1) -- (A.in 2);
    \draw (A.in 1 |- A.out) to (-.75, .35);
    \draw (B.in 1) -- (B.in 2);
    \draw (B.in 1 |- B.out) to (-.75, -.35);
    \draw (A.out) -| (C.in 1);
    \draw (B.out) -| (C.in 2);
\end{circuitikz}

\begin{circuitikz}
    % gates
    \node (text) at (-4.725, 0) {NOR as OR};
    \node[ieeestd nor port, number inputs=2, scale=0.5] (A) at (-1.5, 0) {};
    \node[ieeestd nor port, number inputs=2, scale=0.5] (B) at (0, 0) {};

    % nor inputs
    \draw (B.in 1) -- (B.in 2);
    \draw (B.in 1 |- B.out) to (-.75, 0) to (A.out);

    % nor inputs
    \node[left] at (A.in 1) {};
    \node[left] at (A.in 2) {};
\end{circuitikz}

\begin{circuitikz}
    % gates
    \node (text) at (-3.125, 0) {NOR as NAND};
    \node[ieeestd nor port, number inputs=2, scale=0.5] (A) at (0, .35) {};
    \node[ieeestd nor port, number inputs=2, scale=0.5] (B) at (0, -.35) {};
    \node[ieeestd nor port, number inputs=2, scale=0.5] (C) at (1.5, 0) {};
    \node[ieeestd nor port, number inputs=2, scale=0.5] (E) at (3, 0) {};

    % nor inputs
    \draw (A.in 1) -- (A.in 2);
    \draw (A.in 1 |- A.out) to (-.75, .35);
    \draw (B.in 1) -- (B.in 2);
    \draw (B.in 1 |- B.out) to (-.75, -.35);
    \draw (A.out) -| (C.in 1);
    \draw (B.out) -| (C.in 2);
    \draw (E.in 1) -- (E.in 2);
    \draw (E.in 1 |- E.out) to (C.out);
\end{circuitikz}

\begin{figure}[h]
    NOR as XNOR \hspace{2em} \includegraphics[width=0.5\columnwidth]{Figures/xnor_from_nor.png}
\end{figure}

\begin{figure}[h!]
    NOR as XOR \hspace{2em} \includegraphics[width=0.65\columnwidth]{Figures/xor_from_nor.png}    
\end{figure}

\vspace{-.5em}
\dotfill

\textbf{Miscellaneous Concepts}

\hspace{-1em}
\begin{circuitikz}
    % gates
    \node[ieeestd or port, number inputs=2, scale=0.5] (A) at (0, 0) {};
    \node[ieeestd nand port, number inputs=2, scale=0.5] (B) at (2, 0) {};
    \node (equiv) at (1, 0) {$\equiv$};

    % inverter on OR
    \node[notcirc, right] at (A.in 1) {};
    \node[notcirc, right] at (A.in 2) {};
    \node[left] at ([xshift=-1.5mm]A.in 1) (or1) {};
    \node[left] at ([xshift=-1.5mm]A.in 2) (or2) {};

    % increasing tails on AND
    \draw (or1) -- (A.in 1);
    \draw (or2) -- (A.in 2);

    % inverter on AND
\end{circuitikz}    
\hfill
\begin{circuitikz}
    % gates
    \node[ieeestd and port, number inputs=2, scale=0.5] (A) at (0, 0) {};
    \node[ieeestd nor port, number inputs=2, scale=0.5] (B) at (2, 0) {};
    \node (equiv) at (1, 0) {$\equiv$};

    % inverter on AND
    \node[notcirc, right] at (A.in 1) {};
    \node[notcirc, right] at (A.in 2) {};
    \node[left] at ([xshift=-1.5mm]A.in 1) (or1) {};
    \node[left] at ([xshift=-1.5mm]A.in 2) (or2) {};

    % increasing tails on AND
    \draw (or1) -- (A.in 1);
    \draw (or2) -- (A.in 2);
\end{circuitikz}

Transistor count \vspace{-1em}
\begin{itemize}
    \item NOT: 2
    \item NAND, NOR: $2 \times \text{inputs}$ \hfill AND = NAND + NOT
    \item AND, OR: $2 \times \text{inputs} + 2$ \hfill OR = NOR + NOT
\end{itemize} \vspace{-1em}

Min transistor count: $2 \times \text{inputs} + 2 \times \text{inverters}$


\cleardoublepage


\textbf{Karnaugh Maps}

Implicant: groups of 1s on map in powers of 2 and is rectangular.

Prime implicant: largest possible implicant.

Essential prime implicant: prime implicant that contains at least 1 term not covered by another prime implicant (can overlap).

\vspace{-1em}
\begin{figure}[ht]
    \centering
    \begin{karnaugh-map}[2][4][1][][]
        \manualterms{1, 0, X, 0, 0, X, 1, 1}
        \implicant{0}{2}
        \implicant{6}{7}

        % labels
        \draw[color=black, ultra thin] (0, 4) --
        node [pos=0.7, above right, anchor=south west] {$a$}
        node [pos=0.7, below left, anchor=north east] {$bc$}
        ++(135:1);
    \end{karnaugh-map}
\end{figure}

\vspace{-4.5em}
\begin{align*}
    f &= \sum m(0, 3, 7) + d(1, 6) = a'b' + bc \\
    &= \prod M(2, 4, 5) + d(1, 6) = (a' + b)(b' + c)
\end{align*} \vspace{-2em}

\vspace{-.5em}
\dotfill


\textbf{Adders and Subtractors}

\textit{Half Adder} \vspace{-.5em}

\begin{minipage}{0.6\columnwidth}
    \begin{circuitikz}
        \node[ieeestd xor port, scale=0.65] (xorone) at (0, 1.25) {};
        \node[ieeestd and port, scale=0.65] (and) at (0, 0) {};
        \node[left=0.5cm] (a) at (xorone.in 1) {A};
        \node[left=0.5cm] (b) at (xorone.in 2) {B};

        \draw (a.east) to[short,-*] (xorone.in 1) |- (and.in 1);
        \draw (b.east) to[short,-*] ($(b.east)!.5!(xorone.in 2)$) coordinate (branch) -- (xorone.in 2);
        \draw (branch) |- (and.in 2);

        % labels
        \node[right] at (xorone.out) {$S = A \oplus B$};
        \node[right] at (and.out) {$C_0 = AB$};
    \end{circuitikz}
\end{minipage}
\begin{minipage}{0.35\columnwidth}
    \begin{tabular}{c c | c c}
        A & B & $S$ & $C_0$ \\
        \hline
        0 & 0 & 0 & 0 \\
        0 & 1 & 1 & 0 \\
        1 & 0 & 1 & 0 \\
        1 & 1 & 0 & 1 \\
    \end{tabular}
\end{minipage}

\textit{Full Adder}

\begin{circuitikz}  
    \draw (0,2) node (A) [ieeestd xor port, anchor=in 1, scale=0.65] {};
    \node [ieeestd and port, anchor=in 1, scale=0.65] (B) at (0,0) {};

    \draw   (A.in 1) to [short,-*] ++ (-0.50,0) coordinate (Bin)
    |- (B.in 1)
    (A.in 2)    to [short,-*]   ++  (-0.75,0)   coordinate (Cin)
    |- (B.in 2)
    (Bin)       to [short]      ++  (-0.75,0)    node[left] {B}
    (Cin)       to [short]      ++  (-0.50,0)    node[left] {C\textsubscript{in}}

    (A.out) to [short,*-] ++ (+0.75,0)

    node (D) [ieeestd xor port, anchor=in 2, scale=0.65] {}
    (D.in 1) to [short,-*] ++ (-0.50,0) coordinate (Din)
    |- (Bin |- D.north) -- ++ (-0.75,0) node[left] {A}

    (D.in 2) ++ (0,-1)  node (E) [ieeestd and port,  anchor=in 1, scale=0.65] {}
    (D.out);

    % wiring
    \draw (A.out) |- (E.in 2)(Din) |-  (E.in 1)
    (E.out |- B.out) node (F) [ieeestd or port, anchor=in 2, scale=0.65] {} 
    (B.out) --  (F.in 2)
    (E.out) --  (F.in 1);

    % labels
    \node[right] at (D.out) {$S = A \oplus B \oplus C_{\text{in}}$};
    \node[right] at (F.out) {C\textsubscript{out}};
\end{circuitikz}

C$_{\text{out}} = AB + AC_{\text{in}} + BC_{\text{in}} = AB + C_{\text{in}}(A \oplus B)$

\begin{table}[ht]
    \centering
    \begin{tabular}{c c c | c c}
        A & B & C$_{\text{in}}$ & $S$ & C$_{\text{out}}$ \\
        \hline
        0 & 0 & 0 & 0 & 0 \\
        0 & 0 & 1 & 1 & 0 \\
        0 & 1 & 0 & 1 & 0 \\
        0 & 1 & 1 & 0 & 1 \\
        1 & 0 & 0 & 1 & 0 \\
        1 & 0 & 1 & 0 & 1 \\
        1 & 1 & 0 & 0 & 1 \\
        1 & 1 & 1 & 1 & 1 \\
    \end{tabular}
\end{table}

\textit{Half-Subtractor} \vspace{-.5em}

\begin{minipage}{0.6\columnwidth}
    \begin{circuitikz}
        \node[ieeestd xor port, scale=0.65] (xorone) at (0, 1.25) {};
        \node[ieeestd and port, scale=0.65] (and) at (0, 0) {};
        \node[left=0.5cm] (a) at (xorone.in 1) {A};
        \node[left=0.5cm] (b) at (xorone.in 2) {B};

        \draw (a.east) to[short,-*] (xorone.in 1) |- (and.in 1);
        \draw (b.east) to[short,-*] ($(b.east)!.5!(xorone.in 2)$) coordinate (branch) -- (xorone.in 2);
        \draw (branch) |- (and.in 2);
        \node[notcirc, scale=0.65, xshift=0.275cm] at (and.in 1) {};

        % labels
        \node[right] at (xorone.out) {$D = A \oplus B$};
        \node[right] at (and.out) {$B_0 = A'B$};
    \end{circuitikz}
\end{minipage}
\begin{minipage}{0.35\columnwidth}
    \begin{tabular}{c c | c c}
        A & B & $D$ & $B_0$ \\
        \hline
        0 & 0 & 0 & 0 \\
        0 & 1 & 1 & 1 \\
        1 & 0 & 1 & 0 \\
        1 & 1 & 0 & 0 \\
    \end{tabular}
\end{minipage}


\newpage


\textit{Full Subtractor}

\begin{circuitikz}
    \draw (0,2) node (A) [ieeestd xor port, scale=0.65, anchor=in 1] {}
    (0,0) node (B) [ieeestd and port, scale=0.65, anchor=in 1] {};
    \node [notcirc, left] at (B.bin 1) {};
    %
    \draw   (A.in 1)    to [short,-*]   ++  (-0.50,0)   coordinate (Bin)
    |- (B.in 1)
    (A.in 2)    to [short,-*]   ++  (-0.75,0)   coordinate (Cin)
    |- (B.in 2)
    (Bin)       to [short]      ++  (-0.75,0)    node[left] {B}
    (Cin)       to [short]      ++  (-0.50,0)    node[left] {B\textsubscript{in}}
    %
    (A.out)     to [short,*-]   ++  (+0.75,0)
    node (D) [ieeestd xor port, scale=0.65, anchor=in 2] {}
    (D.in 1)    to [short,-*]   ++  (-0.50,0)   coordinate (Din)
    |- (Bin |- D.north) -- ++ (-0.75,0) node[left] {A}
    %
    (D.in 2) ++ (0,-1)  node (E) [ieeestd and port, scale=0.65,  anchor=in 1] {}
    (D.out)     node[right] {$D = A \oplus B \oplus C_{\text{in}}$};
    \node [notcirc, left] at (E.bin 1) {};
    %
    \draw   (A.out)     |-  (E.in 2)(Din)   |-  (E.in 1)
    (E.out |- B.out) node (F) [ieeestd or port, scale=0.65,anchor=in 2] {} 
    (F.out) node [right] {B\textsubscript{out}}
    (B.out) --  (F.in 2)
    (E.out) --  (F.in 1);
\end{circuitikz}

B$_{\text{out}} = A'B + A'B_{\text{in}} + BB_{\text{in}}$

\begin{table}[ht]
    \centering
    \begin{tabular}{c c c | c c}
        A & B & B$_{\text{in}}$ & $D$ & B$_{\text{out}}$ \\
        \hline
        0 & 0 & 0 & 0 & 0 \\
        0 & 0 & 1 & 1 & 1 \\
        0 & 1 & 0 & 1 & 1 \\
        0 & 1 & 1 & 0 & 1 \\
        1 & 0 & 0 & 1 & 0 \\
        1 & 0 & 1 & 0 & 0 \\
        1 & 1 & 0 & 0 & 0 \\
        1 & 1 & 1 & 1 & 1 \\
    \end{tabular}
\end{table}


\textit{Parallel Binary Adder}

\includegraphics[width=\columnwidth]{Figures/4-bit-adder.jpeg}

\textit{Parallel Binary Subtractor}

\includegraphics[width=\columnwidth]{Figures/4-bit-Subtractor.jpeg}


\textit{Adder and Subtractor}

M is control bit. If subtracting, M adds 1 and inverts all B bits. If adding, M is 0 (i.e., does nothing).

\includegraphics[width=\columnwidth]{Figures/addsub.jpeg}


\cleardoublepage


\textbf{Design of Multi-Level NAND- and NOR-Gate Circuits}

\vspace{-1em}
\begin{enumerate}
    \item Design multi-lvl circuit w/ AND and OR gates.
    \begin{itemize}
        \item Output must be an OR gate.
        \item Each lvl must alternate btw AND and OR.
    \end{itemize}
    \item Number all lvls w/ output as lvl 1. Replace all gates w/ NAND gates.
    \begin{itemize}
        \item Leave inputs to even lvls unchanged.
        \item Invert literal inputs to odd lvls.
    \end{itemize}
\end{enumerate} \vspace{-1em}

Procedure for NOR same except (i) output is AND and (ii) all gates replaced w/ NOR gates.


\end{document}
