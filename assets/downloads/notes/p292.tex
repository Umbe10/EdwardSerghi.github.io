\documentclass[twocolumn]{article}
\usepackage[margin=1.9cm]{geometry}
\usepackage{hyperref}
\hypersetup{
    colorlinks,
    citecolor=blue,
    filecolor=blue,
    linkcolor=blue,
    urlcolor=blue
}

\usepackage{graphicx, amsmath, amsfonts, multicol, color, listings, siunitx}
\setlength{\parindent}{0pt}

\definecolor{dkgreen}{rgb}{0,0.6,0}
\definecolor{gray}{rgb}{0.5,0.5,0.5}
\definecolor{mauve}{rgb}{0.58,0,0.82}

\lstset{frame=tb,
  language=Java,
  aboveskip=3mm,
  belowskip=3mm,
  showstringspaces=false,
  columns=flexible,
  basicstyle={\small\ttfamily},
  numbers=none,
  numberstyle=\tiny\color{gray},
  keywordstyle=\color{blue},
  commentstyle=\color{dkgreen},
  stringstyle=\color{mauve},
  breaklines=true,
  breakatwhitespace=true,
  tabsize=3
}

\usepackage[labelfont=bf, tableposition=top]{caption}
% \captionsetup[table]{skip=1em}
\def\arraystretch{1.15}

\title{PHYS 292 Notes}
\author{Eddie Guo\\University of Alberta}
\date{\today}


\begin{document}
\maketitle

% DARTBOARD STATISTICS
\section{Dartboard Statistics}
\subsection{Statistics}
\subsubsection{Uncertainty Measurements}
\begin{itemize}
    \item Assertions of uncertainties w/o some explanation of how they were estimated are \underline{useless}.
    \item If $t=2.4$ s, report it as $t = 2.4 \pm 0.1$ s.
    \begin{itemize}
        \item Uncertainties should almost always have 1 (or sometimes 2) significant figures.
        \item Measured value should not be more accurate than the uncertainty (i.e., to same digit or decimal place as the standard error).
    \end{itemize}
    \item Measurements \textit{accurate} when \textit{random} uncertainties reduced to an acceptable lvl.
    \item Measurements \textit{precise} when \textit{systematic} uncertainties reduced to an acceptable lvl.
\end{itemize}

\subsubsection{Gaussian PDF}
$\displaystyle G(x, \bar{x}, \sigma) = \frac{1}{\sigma \sqrt{2\pi}} \exp \left[ -\frac{(x-\bar{x})^2}{2 \sigma^2} \right]$
\begin{itemize}
    \item Normalization: $\displaystyle \int_{-\infty}^\infty G(x, \bar{x}, \sigma)\ dx = 1$
    \item Mean: $\displaystyle \int_{-\infty}^\infty x G(x, \bar{x}, \sigma)\ dx = 1$
    \item Variance: $\displaystyle \int_{-\infty}^\infty (x-\bar{x})^2 G(x, \bar{x}, \sigma)\ dx = 1$
\end{itemize}

\subsubsection{Other Impt Equations} \vspace{-1em}
\begin{align*}
    \bar{x} &= \mathbb{E}(X) = \frac{\sum x_i}{n} \\
    \sigma_{N-1} &= \sqrt{\frac{1}{N-1} \sum(x_i - \bar{x})^2} \\
    \alpha &= \frac{\sigma_{N-1}}{\sqrt{N}}
\end{align*}


\subsection{Lab}
\begin{itemize}
    \item Testing random dist'n of errors using darts.
    \item This lab is basically free marks so don't mess up lol.
\end{itemize}

% CONSTANT-V GAS THERMOMETER
\section{Const-V Gas Thermometer}

\subsection{Statistics}
\subsubsection{Ordinary Least Squares (OLS)} \label{subsec:ols} \vspace{-1em}
\begin{align*}
    y &= A + Bx\\
    P_{A,B}(y_i) &\propto \frac{1}{\sigma_y} e^{(y_i A - Bx_i)^2/(2 \sigma_y^2)} \\
    P_{A,B} (y_i) &= \prod_{i=1}^N P_{A,B}(y_i) \propto \frac{1}{\sigma_y^N} e^{-\chi^2/2} \\
    \intertext{where $\chi^2 = \sum_{i=1}^N \frac{(y_i-A-B_{x_i})^2}{\sigma_i^2}$. We can minimize $\chi^2$ via partial derivatives wrt $A$ and $B$ to obtain:}
    &\begin{cases}
        AN + B \sum x_i = \sum y_i \\
        A \sum{x_i} + B \sum x_i^2 = \sum x_i y_i
    \end{cases}
    \intertext{And the solution to this system is:}
    A &= \frac{(\sum x^2 \sum y - \sum x \sum xy)}{\Delta} \\
    B &= \frac{N \sum xy - \sum x \sum y}{\Delta} \\
    \Delta &= N \sum x^2 - (\sum x)^2
    \intertext{And the errors are:}
    \sigma_y &= \sqrt{\frac{1}{N-2} \sum(y_i - A - Bx_i)^2} \\
    \sigma_A &= \sigma_y \sqrt{\frac{\sum x^2}{\Delta}}, \quad\quad\quad \sigma_B = \sigma_y \sqrt{\frac{N}{\Delta}}
\end{align*}


\subsection{Lab}

\begin{itemize}
    \item Goals:
    \begin{itemize}
        \item Det abs zero temp using ideal gas.
        \item Measure P using MEMS (microelectromechanical system).
        \item Perform lin reg using least squares.
        \item Test if model (ideal gas) can explain expt.
    \end{itemize}
    \item Procedure:
    \begin{itemize}
        \item Make P measruements of a const V.
        \item Anallyze data using ideal gas model.
        \item Obtain abs zero temp.
    \end{itemize}
    \item $PV = nRT$
    \item Assume gas behaves like billiard balls w/ elastic collisions. Low density, no interactions outside collisions.
    \item Ctrl $T$ and $V$ to determine $P$.
    \item Gas trapped in bulb, heating water w/ bunsen burner. $P$ measured w/ Raspberry Pi sensor (Honeywell TruStability$^\text{\textregistered}$ Board Mount Pressure Sensors). Works via piezoelectrics.
\end{itemize}


% E/M Ratio
\section{E/M Ratio}

\subsection{Statistics}
If variables indep,
\begin{equation*}
    \sigma_f^2 = \sum_i \left( \frac{\partial f}{\partial x_i} \sigma_i \right)^2
\end{equation*}

If not indep, add covariance terms (linear approx):
\begin{equation*}
    \sigma_f^2 = \sum_i \left( \frac{\partial f}{\partial x_i} \sigma_i \right)^2 + \mathop{\sum \sum}_{i \neq j} \left( \frac{\partial f}{\partial x_i} \right) \left( \frac{\partial f}{\partial x_j} \right) \sigma_{ij}
\end{equation*}

% In its full glory:
% \begin{align*}
%     C_Y &= F_X C_X F_X^T
%     \intertext{where}
%     F_X &= [\nabla_X \cdot f(X)^T]^T \\
%     C_X &= \begin{bmatrix}
%         \sigma^2_{X_1} & \sigma_{X_1, X_2} & \cdots & \sigma_{X_1, X_n} \\
%         \sigma^2_{X_2, X_1} & \sigma_{X_2}^2 & \cdots & \sigma_{X_2, X_n} \\
%         \vdots & \vdots & \ddots & \vdots \\
%         \sigma^2_{X_n X_1} & \sigma_{X_n, X_2} & \cdots & \sigma_{X_n}^2 \\
%     \end{bmatrix} \\
%     \intertext{For 2 variables,}
%     C_Y &= \begin{bmatrix}
%         \sigma_{Y_1}^2 & \sigma_{Y_1, Y_2} \\
%         \sigma_{Y_2, Y_1} & \sigma_{Y_2}^2
%     \end{bmatrix} \\
% \end{align*}

\subsection{Lab}
\begin{itemize}
    \item Goals
    \begin{itemize}
        \item Bending of charges in \textbf{B} to est their E.
        \item Analyze data from CRT.
        \item Perform a linear fit, extract the e/m ratio from the slope of the fit.
    \end{itemize}
    \item Use WLS (NOT OLS) for this experiment.
    \item R, v, and B relation: $mv = qBR$
    \item Energy relation: $E_k = \frac{1}{2}mv^2 = eV_A$
    \item Helmholtz coil B strength: $B_H = \left( \frac{4}{5} \right)^{3/2} \frac{\mu_0 NI_H}{R}$
    \item The ratio e/m (or m/e) will be the slope of your fn.
    \item Vary electron kinetic energy and magnetic fields and record the bending radius of the cathode rays.
    \item  Modify the settings to get the electrons to pass over each of the crossbars: which are placed at a known radius. You can:
    \begin{itemize}
        \item Change V while keeping B constant,
        \item Change B while keeping V constant, and
        \item Change both B and V simultaneously.
    \end{itemize}
\end{itemize}


% Millikin's Oil Drop
\section{Millikan's Oil Drop}

\subsection{Statistics}
Consider the following cases for uncertainties in linear least squares fitting:
\begin{enumerate}
    \item Error in $y$, no error in $x$: just use $\sigma_y$.
    \item Error in $x$, no error in $y$: $\sigma_y(\text{equiv}) = \frac{dy}{dx} \sigma_x = B \sigma_x$
    \item Error in $x$ and $y$: $\sigma_y(\text{equiv}) = \sqrt{\sigma_y^2 + (B \sigma_x)^2}$
\end{enumerate}

Remarks:
\begin{itemize}
    \item $\sigma_y$ and $B$ computed using eqs in Section (\ref{subsec:ols}).
    \item Use $dy/dx$ instead of $B$ for non-linear least squares.
    \item Use a weighted fit if errors do not have the same uncertainty: $w_i = 1/\sigma_i^2$, where $\sigma_i$ is the associated error for each $y_i$:
    \begin{align*}
        A &= \frac{\sum wx^2 \sum wy - \sum wx \sum wxy}{\Delta} \\
        B &= \frac{\sum w \sum wxy - \sum wx \sum wy}{\Delta} \\
        \Delta &= \sum w \sum wx^2 - (\sum wx)^2\\
        w_i &= \frac{1}{\sigma_i^2} \\
        \sigma_A &= \sqrt{\frac{\sum wx^2}{\Delta}}, \quad\quad\quad \sigma_B = \sqrt{\frac{\sum w}{\Delta}}
    \end{align*}
\end{itemize}

\subsection{Lab}
\begin{itemize}
    \item Objectives:
    \begin{itemize}
        \item Measure charge of electrons.
        \item Experimentally est charge quantization.
    \end{itemize}
    \item Experimental goals:
    \begin{itemize}
        \item Use terminal velocities to est forces and charge.
        \item Analyze data from droplets falling and rising uner diff forces.
        \item Search for evidence of charge quantization.
    \end{itemize}
    \item Procedure: track oil droplets in controlled setup. Measure their velocities, analyze the data.
    \item Millikan's expt showed:
    \begin{itemize}
        \item Could suspend single oil drops using $\mathbf{E}$ to balance gravity.
        \item Measured terminal velocity of oil drop to det its radius. Knowing the oil's density, we can find itss mass.
        \item He found the charge each drop to be a multiple of $e =1.601 \times 10^{-19}$ C.
    \end{itemize}
    \item $a = 0 \Longrightarrow mg = kv_f$
    \item If each droplet has a charge, $qE = mg + kv_f$.
    \item Sphere moving in fluid: $kv_f = 6 \pi \eta a^3 \rho$, where $k$ is friction ccoeff, $\eta$ is viscosity, $a$ is radius, $\rho$ is density.
\end{itemize}

% Superconductivity
\section{Superconductivity}
\subsection{Lab}
\begin{itemize}
    \item Goals
    \begin{itemize}
        \item Establish whether a prepared material shows superconductivity.
        \item Search for the Meissner effect.
        \item Measre the resistance of a material.
    \end{itemize}
    \item Meissner effect: expulsin of magnetic field from a superconductr when it's coooled below a critical tempterature, $T_C$.
    \begin{itemize}
        \item Magnetic field lines ``expelled'' from an object when $T < T_C$ (i.e., no field lines pass through the object).
        \item A ssupercurrent induced on surface to exactly cancel the external field within the material.
        \item Requires extremely low temperatures.
    \end{itemize}
\end{itemize}

% Data Rejection
\section{Data Rejection}
Chauvenet's criterion. (If taking more data impractical).
\begin{enumerate}
    \item Assume all data points legitimate, take mean and standard deviation then quantify how anomalous the measurement is by finding its z-score.
    \item Calc probability assoc w/ that z-score for both tails.
    \item Determine number of measurements expected to be that extreme (multiply probability by number of data points collected).
    \item Chauvenet says anything below 0.5 can be rejected.
\end{enumerate}
Alternatively, obtain final results w/ and w/o rejection.


% PHYS 292B notes begin here
\cleardoublepage

\section{Franck-Hertz \& Arbitrary Fits}
\subsection{Statistics}
\subsubsection{Fitting an Arbitrary Function}
\begin{itemize}
    \item In Franck-Hertz, valleys look like quadratic functions: $y = a(x+b)^2 + c$. We want to compute $b$.
\end{itemize} \vspace{-.5em}

\begin{lstlisting}[language=python]
import numpy as np
from scipy import optimize

def myquad(x, a, b, c):
    return a*(x-b)**2+c

# non-linear least squares
# popt = estimated parameters
# pcov = covariance matrix
popt, pcov = optimize.curve_fit(
    myquad, x, y,
    [100, 7, 50],  # initial guess
    sigma=yerr)

perr = np.sqrt(np.diag(pcov))
\end{lstlisting}

\subsection{Lab}
\begin{itemize}
    \item Defend atomic model w/ E lvls using Franck-Hertz expt (think Bohr and quantization).
    \item Goals:
    \begin{itemize}
        \item Operate vacuum tube w/ hot cathode inside.
        \item Measure small currents w/ an electrometer.
        \item Est excitation E of Hg.
    \end{itemize}
    \item Franck-Hertz
    \begin{enumerate}
        \item Accelerate electrons to some known E (use parallel plate capacitor).
        \item Now place collector plate past the accelerating voltage. Put negative $V_s$ on collector plate. Only e$^-$ w/ $T > V_s$ will hit (stopping voltage).
        \item Continue increasing $V_s$ until e$^-$ start exciting Hg atoms and measure current.
        \item Once an Hg atom is excited, current drops suddenly then increases again (peaks and valleys in $I$ vs $V_s$ plot). Diff btw peaks and valleys is the E diff for raising Hg an energy state.
    \end{enumerate}
\end{itemize}

\section{Radioactivity of Radon \& Exponential Fits}
\subsection{Lab}
\begin{itemize}
    \item Understand and measure radioactive decays using radon.
    \item Goals:
    \begin{itemize}
        \item Use leaf electroscope to collect charges.
        \item Take differential measurement vs time.
        \item Fit the half-life of radon-220.
    \end{itemize}
    \item Radioactive decay relates to stability in binding the nucleus together.
    \item 3 types of decays:
    \begin{itemize}
        \item $\alpha$: Emission of an He nucleus (2p2n). Paper can stop it (thin).
        \item $\beta$: Neutron to electron + neutrino. Thin sheet of metal can stop it (normal).
        \item $\gamma$: Change in internal E config leading to emission of $\gamma$ ray. Lead shielding can stop it (thick).
    \end{itemize}
    \item Decay is a quantum mechanical effect.
    \item Decay rate: $\frac{dN_n}{dt} = -kN_n \implies N_n(t) = N_0 e^{-kt}$
    \subitem Half-life: $t_{1/2} = \ln 2 / k, \quad k$ is prop const
    \item Th-232 has a long decay series.
\end{itemize}

\subsection{Statistics}
\begin{itemize}
    \item Linear fit: simple, can be done by hand.
    \item Arbitrary curve fit: requires a computer.
    \item Exponential functions can be linearized.
    \subitem $N_n = N_0 e^{-kt} \implies \ln N_n = \ln N_0 - kt$
\end{itemize}


\section{Faraday Effect \& Goodness of Fit}
\subsection{Statistics}
\begin{itemize}
    \item Objectives: Corroborate Malus' law and explore Faraday rotation.
    \item Goals:
    \begin{itemize}
        \item Explore polarization of light.
        \item Learn how to measure small signals.
    \end{itemize}
    \item Procedure
    \begin{itemize}
        \item Measure intensity of polarized beam.
        \item Pass beam thru dielectric material.
        \item Pass beam thru material in magnetic field.
    \end{itemize}
    \item 3 types of polarized light: linear (plane waves), circular (\SI{90}{\degree} phase diff), elliptical (2 plane waves of diff amplitude but relative phase shift of \SI{90}{\degree}).
    \begin{itemize}
        \item Linear pol: $\mathbf{E}_l = E_0 \cos(kz - \omega t + \phi_1) \hat{x}$
        \item Circular pol: $\mathbf{E}_c = E_0 \cos(kz - \omega t + \phi_2) \hat{x} - E_0 \sin(kz - \omega t + \phi_2) \hat{y}$
        \item Elliptical pol: $E_l(x, y, 0) \hat{x} + E_l(x, y, \frac{-\pi}{2}) \hat{y}$
    \end{itemize}
    \item Can make linearly polarized light by expressing it as circularly polarized beams: $\mathbf{E}_l = \mathbf{E}_c (x, y, \phi) + \mathbf{E}_c(x, y, -\phi)$
    \item Light interacts w/ materials (light can't interact w/ light due to superposition), so can polarize beam of unpolarized light by decoupling their components.
    \item $\Delta \theta = VBL$, $V$ is Verdet's constant (how strongly $B$ affects refractive index), $L$ is length of medium.
    \item Malus' law: $I = I_0 \cos^2 \theta$, since $V \propto I \propto E^2$ then $V = V_0 \cos^2(\theta - \theta_0)$
\end{itemize}

\subsection{Lab}
Recall: $\chi^2 = \sum_{i=1}^N \frac{(y_i - A - Bx_i)^2}{\sigma_i^2}$. Let's generalize this:
\begin{equation*}
    \chi^2 = \sum_{i=1}^N \frac{(y_i - y_{m,i})^2}{\sigma_i^2}
\end{equation*}
where $y_{m,i}$ is the model or expected value. This parameter can be used to test the agreement btw a dataset and a model. \\

Degrees of freedom: $\nu = N_{\text{data}} - N_{\text{params}}$. Goodness of fit can be estimated using the reduced chi-squared:
\begin{equation}
    \chi_\nu^2 = \left( \sum_{i=1}^N \frac{(y_i - y_{m,i})^2}{\sigma_i^2} \right) / \nu
\end{equation}
Naively, expect each dof contributes 1 unit, dist'n.
\begin{itemize}
    \item $\chi_\nu^2 = 0 \implies$ no uncertainty and perfect fit w/ dist'n.
    \item $\chi_\nu^2 = 1 \implies$ data well described by proposed dist'n.
    \item $\chi_\nu^2 \gg 1 \implies$ unlikely data comes from proposed dist'n.
\end{itemize}
Can compute p-value via integration of pdf of the chi-squared dist'n w/ a given dof.

\begin{lstlisting}[language=python]
import numpy as np
from scipy import stats

# assume variables exist
chi_sq = np.sum((y_exp - y_model)**2 / error**2)
dof = x.length() - 1
reduced_chi_sq = chi_sq / dof
p = 1 - stats.chi2.cdf(chi_sq, dof)
print(f'p-value: {p}')
\end{lstlisting}


\section{Covariance \& Correlation}
\subsection{Derivation of Covariance}
\begin{align*}
    q_i &= q(x_i, y_i) \\
    q_i &\approx q(\bar{x}, \bar{y}) + \frac{\partial q}{\partial x} (x_i - \bar{x}) + \frac{\partial q}{\partial y} (y_i - \bar{y}) \\
    \bar{q} &= \frac{\sum_i q_i}{N} = q(\bar{x}, \bar{y})\\
    &= \frac{1}{N} \sum_i \left( q(\bar{x}, \bar{y}) + \frac{\partial q}{\partial x} (x_i - \bar{x}) + \frac{\partial q}{\partial y} (y_i - \bar{y}) \right) \\
    \sigma^2_q &= \frac{1}{N} \sum_i (q_i - \bar{q})^2 \\
    \sigma_q^2 &= \left( \frac{\partial q}{\partial x} \right)^2 \underbrace{\frac{1}{N} \sum_i (x_i - \bar{x})^2}_{\sigma_x^2} + \left( \frac{\partial q}{\partial y} \right)^2 \underbrace{\frac{1}{N} \sum_i (y_i - \bar{y})^2}_{\sigma_y^2} \\
    &\quad\quad + 2 \frac{\partial q}{\partial x} \frac{\partial q}{\partial y} \underbrace{\frac{1}{N} \sum_i (x_i - \bar{x}) (y_i - \bar{y})}_{\sigma_{xy}} \\
    & = \left( \frac{\partial q}{\partial x} \right)^2 \sigma_x^2 + \left( \frac{\partial q}{\partial y} \right)^2 \sigma_y^2 + 2 \frac{\partial q}{\partial x} \frac{\partial q}{\partial y} \sigma_{xy}
\end{align*}

\subsection{Remarks}
\begin{itemize}
    \item Covariance is $\sigma_{xy} = \frac{1}{N} \sum_i (x_i - \bar{x}_i) (y_i - \bar{y})$.
    \item If $\sigma_{xy} \neq 0$, errors of $x$ and $y$ are correlated.
    \item Schwarz's inequality: $|\sigma_{xy}| \leq \sigma_x \sigma_y \implies \sigma_q \leq \left| \frac{\delta q}{\delta x} \right| \sigma_x + \left| \frac{\delta q}{\delta y} \right| \sigma_y$
    \item If you have more variables, $\sigma_{xy}$ becomes a matrix.
\end{itemize}

\subsection{Correlation}

\begin{equation}
    r = \frac{\sigma_{xy}}{\sigma_x \sigma_y}
\end{equation}


\end{document}
