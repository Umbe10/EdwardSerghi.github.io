\documentclass{article}
\usepackage[utf8]{inputenc}
\usepackage{multicol}
\usepackage[utf8]{inputenc}
\usepackage[T1]{fontenc}
\usepackage[margin=1in]{geometry}
\usepackage{hyperref}
\hypersetup{
    colorlinks,
    citecolor=blue,
    filecolor=blue,
    linkcolor=blue,
    urlcolor=blue
}
\usepackage{graphicx}
\usepackage{mathtools}
\usepackage{soul}
\usepackage{csvsimple}
\usepackage{hanging}
\usepackage{multicol}
\usepackage{amsmath}
\usepackage{listings}
\usepackage{amssymb}
\usepackage{float}
\usepackage{longtable}
\usepackage{pgfplotstable,filecontents}
\pgfplotsset{compat=1.9}
\usepackage[compact]{titlesec}
\titlespacing{\section}{0pt}{*0}{*0}
\titlespacing{\subsection}{0pt}{*0}{*0}
\titlespacing{\subsubsection}{0pt}{*0}{*0}
\usepackage{xcolor}
\usepackage{enumitem}
 
\definecolor{codegreen}{rgb}{0,0.6,0}
\definecolor{codegray}{rgb}{0.5,0.5,0.5}
\definecolor{codepurple}{rgb}{0.58,0,0.82}
\definecolor{backcolour}{rgb}{0.95,0.95,0.92}
 
\lstdefinestyle{mystyle}{
    backgroundcolor=\color{backcolour},   
    commentstyle=\color{codegreen},
    keywordstyle=\color{magenta},
    numberstyle=\tiny\color{codegray},
    stringstyle=\color{codepurple},
    basicstyle=\ttfamily\footnotesize,
    breakatwhitespace=false,         
    breaklines=true,                 
    captionpos=b,                    
    keepspaces=true,                 
    numbers=left,                    
    numbersep=5pt,                  
    showspaces=false,                
    showstringspaces=false,
    showtabs=false,                  
    tabsize=2
}
 
\lstset{style=mystyle}

\setlength\parindent{0pt}

\frenchspacing

\title{Python 3 Exceptions \& Handling Errors}
\author{Eddie Guo}
\date{October 2019}


\begin{document}
\lstset{language=Python}
\maketitle

% Introduction to Recursion
\section{Introduction to Exceptions}
% Topics Covered
\subsection{Topics Covered}
    \begin{multicols}{2}
        \begin{enumerate}[label=(\roman*)]
            \item What are exceptions?
            \item Raising exceptions
            \item Catching exceptions
            \item Assertions
        \end{enumerate}
    \end{multicols}

% What are Exceptions?
\subsection{What are Exceptions?}
    \begin{multicols}{2}
        \begin{itemize}
            \item Input validation != exception handling
            \item Exception: event during execution of program that disrupts normal flow of program
            \item Exceptions allow us to handle errors/exceptional conditions
            \item In Python, exception is obj that reps an error
        \end{itemize}
    \end{multicols}

% Exceptions in Python
\subsection{Exceptions in Python}
    \begin{multicols}{2}
        \begin{itemize}
            \item Python script raises an exception where error detected
            \item Python interpreter raises exception when it detects run-time error
            \item Can explicitly raise an exception
                \begin{itemize}
                    \item '2013' + 1 $\rightarrow$ \textbf{TypeError}: cannot concatenate 'str' and 'int' objects
                    \item 175 + cmput*13 $\rightarrow$ \textbf{NameError}: name 'cmput' is not defined
                    \item 365 * (12/0) $\rightarrow$ \textbf{ZeroDivisionError}: integer division or modulao by zero
                \end{itemize}
            \item Accessing non-existent dictionary key will raise \textbf{KeyError} exception
            \item Searching list for non-existent value will raise \textbf{ValueError} exception
            \item Calling non-existent method will raise \textbf{AttributeError} exceptiont
            \item \href{https://docs.python.org/3/library/exceptions.html}{Python documentation for exceptions}
        \end{itemize}
    \end{multicols}

% Why Use Exceptions?
\subsection{Why Use Exceptions?}
    \begin{itemize}
        \item Separating error-handling code from regular code
        \item Deferring decisions about how to respond to exceptions
        \item Providing mech for specifying diff kinds of exceptions that can arise in program
    \end{itemize}

% Exception Handling Blocks
\subsection{Exception Handling Blocks}
    \begin{multicols}{2}
        \begin{itemize}
            \item If you have code that may raise exception, place code in \texttt{try:} followed by \texttt{except:}
        \item Don't catch exception? Entire program crashes.
        \item \texttt{except} w/o explicit exception will catch all remaining exceptions
        \item \texttt{except} may name none/one/multiple exceptions as parenthesized tuple
\begin{lstlisting}
except (RuntimeError, TypeError, NameError):
        [do something here]
\end{lstlisting}
        \end{itemize}
    \end{multicols}


% Multiple Except Clauses
\subsection{Multiple Except Clauses}
    \begin{multicols}{2}
        \begin{itemize}
            \item \texttt{try} may have >1 \texttt{except} clause to specify handlers for diff exceptions
            \item \hl{At most, one handler will be executed}
            \item Handlers only handle exceptions that occur in corresponding try clause, not in other handlers of same try statement
            \item \hl{Go from specific excpetions to more general ones b/c Python reads top-down}
        \end{itemize}
    \end{multicols}

% The try Statement
\subsection{The \texttt{try} Statement}
    \begin{multicols}{2}
        \begin{itemize}
            \item If no exception raised by code w/in try block (or methods called w/in try block), code executes normally \& all except blocks skipped
            \item If exception arises in try block, execution of try block terminates execution immediately \& \texttt{except} is sought to handle exception
                \begin{enumerate}
                    \item If appropriate except clause found, it's executed
                    \item Elif exception propagated to method or outer \texttt{try} block
                    \item Elif no handler found $\rightarrow$ \textbf{unhandled exception} \& execution stops w/ message
                \end{enumerate}
        \end{itemize}
    \end{multicols}
    
% Propagating Exceptions
\subsection{Propagating Exceptions}
    \begin{multicols}{2}
        \begin{itemize}
            \item An exception will bubble up call stack until it:
                \begin{itemize}
                    \item Reaches method w/ suitable handler or
                    \item Propagates thru main stack (1st method on call stack)
                \end{itemize}
            \item If exception not caught by any method, exception treated like error: stack frames displayed \& program terminates
        \end{itemize}
    \end{multicols}
\vspace{-2em}
\begin{lstlisting}
try:
    f = open('myfile.txt', 'r')
    s = f.readline()
    i = int(s.strip())
except IOError:
    print('File does not exist or cannot be read.')
except ValueError:
    print('Could not convert data to an integer')
except: # If this were first, no IOErrors or ValueErrors will be caught
    print('Unexpected error')
    raise # can explicitly propagate exceptions using raise
\end{lstlisting}

% Raising Exceptions
\subsection{Raising Exceptions}
    \begin{multicols}{2}
        \begin{itemize}
            \item What can be raised as exception?
                \begin{itemize}
                    \item Any standard Python exception
                    \item New instance of exception w/ custom arguments
                    \item Instances of our own specialized exception classes
                \end{itemize}
        \end{itemize}
    \end{multicols}
\vspace{-2em}
\begin{lstlisting}
try:
    print('Raising an exception')
    raise Exception('CMPUT', '274')
except Exception as inst: # the exception instance
    print(inst.args) # arguments stored in .args
    x, y = inst.args # unpacks args
    print('x =', x, 'y =', 'y')
\end{lstlisting}

% else & finally Clause
\subsection{\texttt{else} \& \texttt{finally} Clause}
    \begin{multicols}{2}
        \begin{itemize}
        \item \texttt{else} clause CAN'T come b4 try \& except (i.e., must follow all except clauses)
            \item Code in else clause must be executed if clause does not raise exception
            \item \texttt{finally} will execute regardless if error was raised (executed under ALL circumstances)
            \item \texttt{finally} useful if you wanna perform ``cleanup'' operations b4 exiting method (ex: closing file) \& avoids duplicating code in each \texttt{except} clause
        \end{itemize}
    \end{multicols}
\vspace{-2em}
\begin{lstlisting}
def divide(x, y):
    try:
        result = x / y
    except ZeroDivisionError:
        print('division by zero!')
    else:
        print('the result is', result)
    finally:
        print('thanks for dividing!')
\end{lstlisting}
\vspace{-1em}
\begin{center}
    \includegraphics[scale = 0.45]{Capture.PNG}
\end{center}

% Summary: Possible Execution Paths
\section{Summary: Possible Execution Paths}
    \begin{multicols}{2}
        \begin{enumerate}
            \item No exception occurs
                \begin{enumerate}
                    \item Execute try block
                    \item Execute else \& finally clauses
                    \item Execute rest of method
                \end{enumerate}
            \item Exception occurs \& is caught
                \begin{enumerate}
                    \item Execute try block until 1st exception occurs
                    \item Execute 1st except clause that matches exception
                    \item Execute finally clause
                    \item Execute rest of method
                \end{enumerate}
            \item Exception occurs \& is not caught
                \begin{enumerate}
                    \item Execute try block until 1st exception occurs
                    \item Execute try block until 1st exception occurs
                    \item Execute finally clause
                    \item Propagate exception to calling method
                \end{enumerate}
        \end{enumerate}
    \end{multicols}

% Assertions
\section{Assertions}
        \begin{multicols}{2}
        \begin{itemize}
            \item Assertion is statement that raises \textbf{AssertionError} exception if condition not met
            \item \texttt{assert Expression[, Arguments]}
            \item If assertion fails, Python uses given arg as arg for \texttt{AssertionError}
            \item \texttt{AssertionError} exceptions can be caught \& handled like any other exception
            \item Good practice to place assertions at start of fn to check for valid input, \& after fn call to check for valid output
        \end{itemize}
    \end{multicols}
\vspace{-2em}
\begin{lstlisting}
def KelvintoFahrenheit(temperature):
    assert (temperature >= 0), 'Colder than absolute zero!'
    return ((temperature - 273) * 1.80 + 32

if __name__ == '__main__':
    try:
        fahrenheit = KelvintoFahrenheit(-23)
        print(fahrenheit)
    except AssertionError as my_error:
        print(my_error.args)

# Output
(Colder than absolute zero!, )
\end{lstlisting}

\end{document}
