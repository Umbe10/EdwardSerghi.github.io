\documentclass{article}
\usepackage[utf8]{inputenc}
\usepackage{multicol}
\usepackage[utf8]{inputenc}
\usepackage[T1]{fontenc}
\usepackage[margin=1in]{geometry}
\usepackage{hyperref}
\hypersetup{
    colorlinks,
    citecolor=black,
    filecolor=black,
    linkcolor=black,
    urlcolor=black
}
\usepackage{graphicx}
\usepackage{mathtools}
\usepackage{soul}
\usepackage{csvsimple}
\usepackage{hanging}
\usepackage{multicol}
\usepackage{amsmath}
\usepackage{listings}
\usepackage{amssymb}
\usepackage{float}
\usepackage{longtable}
\usepackage{pgfplotstable,filecontents}
\pgfplotsset{compat=1.9}
\usepackage[compact]{titlesec}
\titlespacing{\section}{0pt}{*0}{*0}
\titlespacing{\subsection}{0pt}{*0}{*0}
\titlespacing{\subsubsection}{0pt}{*0}{*0}
\usepackage{xcolor}
\usepackage{enumitem}
 
\definecolor{codegreen}{rgb}{0,0.6,0}
\definecolor{codegray}{rgb}{0.5,0.5,0.5}
\definecolor{codepurple}{rgb}{0.58,0,0.82}
\definecolor{backcolour}{rgb}{0.95,0.95,0.92}
 
\lstdefinestyle{mystyle}{
    backgroundcolor=\color{backcolour},   
    commentstyle=\color{codegreen},
    keywordstyle=\color{magenta},
    numberstyle=\tiny\color{codegray},
    stringstyle=\color{codepurple},
    basicstyle=\ttfamily\footnotesize,
    breakatwhitespace=false,         
    breaklines=true,                 
    captionpos=b,                    
    keepspaces=true,                 
    numbers=left,                    
    numbersep=5pt,                  
    showspaces=false,                
    showstringspaces=false,
    showtabs=false,                  
    tabsize=2
}
 
\lstset{style=mystyle}

\setlength\parindent{0pt}

\frenchspacing

\title{Python 3 Binary Tree Implementation}
\author{Eddie Guo}
\date{September 2019}


\begin{document}
\lstset{language=Python}
\maketitle

% Introduction to Binary Trees
\section{Introduction to Binary Trees}
% Topics Covered
\subsection{Topics Covered}
    \begin{multicols}{2}
        \begin{enumerate}[label=(\roman*)]
            \item Recursive representation
            \item Tree leaf class
            \item Tree branch class
            \item Binary tree traversal
        \end{enumerate}
    \end{multicols}

% Defining a Binary Tree
\subsection{Defining a Binary Tree}
    \begin{multicols}{2}
        \begin{itemize}
            \item Binary tree made up of 1/more nodes
            \item Each node has 0/1/2 children
            \item All nodes have 1 parent, except root node (which has no parent)
            \item All leaf nodes store a value (byte to be compressed)
            \item All interior nodes are root node of subtree
        \end{itemize}
    \end{multicols}
\begin{center}
    \includegraphics[scale=0.4]{trees.png}
\end{center}

% How to Implement in Python?
\section{How to Implement Binary Trees in Python?}
    \begin{multicols}{2}
        \begin{itemize}
            \item Need way to rep tree nodes \& rels btw them
            \item Option 1: lists of lists
                \begin{itemize}
                    \item Each subtree is list that contains root, left subtree, right subtree
                    \item Gets complicated quickly; hard to keep track of nested subtrees
                \end{itemize}
            \item Option 2: custom classes
                \begin{itemize}
                    \item Tree branch class capable of containing left \& right subtree
                    \item Tree leaf class to rep indiv leaf nodes
                \end{itemize}
        \end{itemize}
    \end{multicols}

\begin{multicols}{2}
% Tree Leaf Class
\subsection{Tree Leaf Class}
        \begin{itemize}
            \item Tree leaf properties: has value (uncompressed byte) that it's storing
            \item Tree leaf behavs: n/a
        \end{itemize}

% Tree Branch (Subtree) Class
\subsection{Tree Branch (Subtree) Class}
    \begin{itemize}
        \item Tree branch properties: left child, right child
        \item Tree branch behavs: n/a
    \end{itemize}
\end{multicols}
\vspace{-2em}
\begin{lstlisting}
class TreeLeaf:
    def __init__(self, uncompressed_byte):
        self.value = uncompressed_byte
    
    def __str__(self):
        return 'Leaf storing: ' + self.value

class TreeBranch:
    def __init__(self, lchild, rchild):
        self.left = lchild
        self.right = rchild
    
    def __str__(self):
        return f'({self.left} <- branch root -> {self.right})'

if __name__ == "__main__":
    leafA = TreeLeaf('a')
    leafB = TreeLeaf('b')
    leafC = TreeLeaf('c')
    print(leafA, leafB)
    branch = TreeBranch(leafA, leafB)
    branch2 = TreeBranch(branch, leafC)
    print(branch)
    print(branch2)

# Output
a b
(a <- branch root -> b)
((a <- branch root -> b) <- branch root -> c)
\end{lstlisting}

% Binary Tree Traversals
\subsection{Binary Tree Traversals}
    \begin{itemize}
        \item Base case is leaf
        \item Processing of left \& right subtrees done recursively
        \item There are 4 common binary tree traversals:
    \end{itemize}
\vspace{-1em}
\begin{center}
    \includegraphics[scale=0.35]{tree_travs.PNG}
    \includegraphics[scale=0.5]{recursion.PNG}
\end{center}

% Binary Tree Traversals: Example
\subsection{Binary Tree Traversals: Example}
    \begin{multicols}{2}
        \begin{itemize}
            \item Preorder: 1 2 3 4 5 6 7 8 9
            \item Inorder: 4 3 5 2 6 8 7 9 1
            \item Postorder: 4 5 3 6 2 8 9 7 1
            \item Levelorder: 1 2 7 3 6 8 9 4 5
        \end{itemize}
    \end{multicols}
\begin{center}
    \includegraphics[scale=0.65]{tree.PNG}
    \includegraphics[scale=0.75]{traversal.PNG}
\end{center}

\end{document}
