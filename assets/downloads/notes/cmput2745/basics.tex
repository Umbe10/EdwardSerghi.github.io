\documentclass{article}
\usepackage[utf8]{inputenc}
\usepackage{multicol}
\usepackage[utf8]{inputenc}
\usepackage[T1]{fontenc}
\usepackage[margin=1in]{geometry}
\usepackage{hyperref}
\hypersetup{
    colorlinks,
    citecolor=black,
    filecolor=black,
    linkcolor=black,
    urlcolor=black
}
\usepackage{graphicx}
\usepackage{mathtools}
\usepackage{soul}
\usepackage{csvsimple}
\usepackage{hanging}
\usepackage{multicol}
\usepackage{amsmath}
\usepackage{listings}
\usepackage{amssymb}
\usepackage{float}
\usepackage{longtable}
\usepackage{pgfplotstable,filecontents}
\pgfplotsset{compat=1.9}
\usepackage[compact]{titlesec}
\titlespacing{\section}{0pt}{*0}{*0}
\titlespacing{\subsection}{0pt}{*0}{*0}
\titlespacing{\subsubsection}{0pt}{*0}{*0}
\usepackage{xcolor}
\usepackage{enumitem}
 
\definecolor{codegreen}{rgb}{0,0.6,0}
\definecolor{codegray}{rgb}{0.5,0.5,0.5}
\definecolor{codepurple}{rgb}{0.58,0,0.82}
\definecolor{backcolour}{rgb}{0.95,0.95,0.92}
 
\lstdefinestyle{mystyle}{
    backgroundcolor=\color{backcolour},   
    commentstyle=\color{codegreen},
    keywordstyle=\color{magenta},
    numberstyle=\tiny\color{codegray},
    stringstyle=\color{codepurple},
    basicstyle=\ttfamily\footnotesize,
    breakatwhitespace=false,         
    breaklines=true,                 
    captionpos=b,                    
    keepspaces=true,                 
    numbers=left,                    
    numbersep=5pt,                  
    showspaces=false,                
    showstringspaces=false,
    showtabs=false,                  
    tabsize=2
}
 
\lstset{style=mystyle}

\setlength\parindent{0pt}

\frenchspacing

\title{Python 3 Basics}
\author{Eddie Guo}
\date{September 2019}

\begin{document}
\lstset{language=Python}

\maketitle


% Topics Covered
\section{Introduction to Python Basics}
% Topics Covered
\subsection{Topics Covered}
\begin{multicols}{2}
    \begin{enumerate}[label=(\roman*)]
        \item Interpreted vs compiled code
        \item Programming style: comments, PEP8
        \item Simple input/output
        \item Values and variables
        \item Introduction to built-in data types \& how to use them
    \end{enumerate}
\end{multicols}

% Intro
\section{Intro}

% Programming = Data + Algorithms
\subsection{Programming = Data + Algorithms}
    Data strucs + pseudo-code algorithms $\rightarrow$  programming lang $\rightarrow$ compiled/interpreted into machine code

    \begin{multicols}{2}
        \begin{itemize}
            \item Why are computers dumb?
                \begin{itemize}
                    \item They take instructions literally.
                \end{itemize}
            \item Why are computers good?
                \begin{itemize}
                    \item B/c they do things over and over rly fast.
                \end{itemize}
            \item Program = set of instrucs given to computer.
            \item Computers understand machine lang (1s \& 0s; i.e., CPU \textbf{only} understands machine lang).
            \item \hl{Unsolvable problems are not computable}
            \item \hl{Programs CANNOT exist w/out algorithms}
        \end{itemize}
    \end{multicols}


    \begin{multicols}{2}
        \textbf{Interpreter (ex: Python 3)}
            \begin{itemize}
                \item Interpreter translates program line-by-line until it meets 1st error/end of program.
                \item Code interpreted every time you run your program.
            \end{itemize}
    
        \textbf{Compiler (ex: C++)}
            \begin{itemize}
                \item Translates entire program into machine code efficiently (execution usually faster).
                \item Code only compiled when new executable req.
            \end{itemize}
    \end{multicols}

% Python Program Style Notes
\subsection{Python Program Style Notes}
    \begin{itemize}
        \item Always include header.
        \item In header, always include what your program does.
        \item Comments improve code readability \& maintainability.
            \begin{itemize}
                \item Should explain approach of code (the 'why,' not line-by-line description).
            \end{itemize}
        \item \hl{To check style of helloworld.py, type \texttt{style helloworld.py} in terminal.}
    \end{itemize}

\begin{lstlisting}[language=Python]
    # ==============================
    #   Name: Eddie Guo
    #   ID: 1576381
    #   CMPUT 274, Fall 2019
    #
    #   Exercise 1: Hello World.
    #   Description here
    # ==============================
\end{lstlisting}
    
% More Python Notes
\subsection{More Python Notes}
    \begin{itemize}
        \item Python is dynamically typed.
            \begin{itemize}
                \item i.e., don't have to explicitly declare variable along w/ type (C++ is diff).
            \end{itemize}
        \item Any var not assoc w/ var is periodically deleted from mem by Python's garbage collector.
    \end{itemize}

% Python Variable Names
\section{Python Variable Names}
    \begin{multicols}{2}
        \begin{itemize}
            \item Python keywords can't be used as var names (ex: and, as, in, class).
            \item \hl{Variable names also called identifier.}
            \item Underscore ex: \texttt{hello\_world}
            \item Lower camel case ex: \texttt{dogsTasteGoodLol}
            \item Upper camel case ex: \texttt{MyNameIsJeeeefffff}
            \item \hl{According to PEP 8, use underscore for multi-word identifiers.}
        \end{itemize}
    \end{multicols}

    \begin{figure}[!h]
        \begin{center}
            \includegraphics[scale = 0.35]{keywords.png}
        \end{center}
    \vspace*{-5mm}
    \caption{Common Python 3 keywords}
    \end{figure}
    
% Built-in Types   
\section{Built-In Types \& Methods}
\hl{\textbf{Immutable means var can't change in place; a new obj is created for each operation. You must assign a variable to ref \& store the new obj. Else, garbage collector will remove.}}
% Buil-In Types int, float, complex
    \subsection{Built-In Types: \texttt{int, float, complex}}
    \begin{multicols}{2}
        \begin{itemize}
            \item \texttt{int, float, complex} are immutable
            \item \texttt{5//2} returns \texttt{2} (floored division)
            \item \texttt{-7//3} returns \texttt{-3} (floored division)
            \item \texttt{5\%2} returns \texttt{1} (modulo operator)
        \end{itemize}
    \end{multicols}

% Convert Type
\subsection{Convert Type}
    \begin{itemize}
        \item Can convert type of one type to another (ex: \texttt{list()}, \texttt{set()}).
        \item Can't mix types when performing operation (ex: \texttt{'CMPUT'+12.0} $\rightarrow$ \texttt{'CMPUT'+str(12.0)}).
    \end{itemize}

% Built-in types: Bool
\subsection{Built-in Types: \texttt{bool}}
    \begin{multicols}{2}
        \begin{itemize}
            \item Boolean is immutable
            \item Remember the truth tables
            \item For 'and', only if both operands \texttt{True}, then True
                \begin{itemize}
                    \item if 1$^{\mathrm{st}}$ item is \texttt{False}, don't eval 2nd part
                \end{itemize}
            \item For 'or', only if both operands \texttt{False}, then false
                \begin{itemize}
                    \item If 1$^{\mathrm{st}}$ item is \texttt{True}, don't eval 2$^{\mathrm{nd}}$ part.
                \end{itemize}
            \item Strings are indexed starting from 0
                \begin{itemize}
                    \item If \texttt{myVar='CMPUT'}, then \texttt{myVar[2]='P'}
                \end{itemize}
            \item \texttt{replace(old, new, max)} method
            \item Rem that \texttt{strip()!=split(char)}
        \end{itemize}
    \end{multicols}
    
    
% format()
\subsection{String Method: \texttt{format()}}
\begin{lstlisting}{Language=Python}
    >>> print('my number is {:15}!'.format(1))
    'my number is 1!'
    
    """
    {:<} - left-justified in field width
    {:^} - centerized in field width
    {:>} - right-justified in field width
    {:015} - pad w/ 0s for field width of 15
    """
    
    >>> print('my number is {:15}!'.format(1))
    >>> print('my number is {:015}!'.format(1))
    my number is               1!
    my number is 000000000000001!
    
    """
    {15.2f} - 2 digits after decimal pnt
    {0} - mapping 1st element in str to 1st argument in format()
    """
    
    name = 'Fred'; amount = 5.43
    >>> print('The person {0:^015} has {1:>07.2f} dollars'.format(name, amount)
    The person 00000Fred000000 has 00005.4 dollars
    """ NOTES:
        - 1st arg in format centerized w/ Fred in middle, field width = 15, empty spaces filled by 0s, 5 0s on left, 6 0s on right
        - 2nd arg in format right-aligned, padded w/ 0s, width = 7, 2 decimal places
        - format() may come in handy for making tables
    """
\end{lstlisting}

% List
\subsection{Built-In Type: \texttt{list}}
    \begin{multicols}{2}
        \begin{itemize}
            \item List is seq of values of \textit{any} type \& is mutable.
            \item Operators \texttt{+}, \texttt{*} concatenate list; \texttt{:} slices lists
                \begin{itemize}
                    \item \texttt{[1,2,3]+[4,5,6]} returns \texttt{[1, 2, 3, 4, 5, 6]}
                    \item \texttt{[1,2,3]*3} returns \texttt{[1, 2, 3, 1, 2, 3, 1, 2, 3]}
                \end{itemize}
            \item \texttt{k=[1,2,3,4,5,6]}
                \begin{itemize}
                    \item \texttt{k[2:3]} returns \texttt{[3, 4]}
                    \item \texttt{k[2:]} returns \texttt{[3, 4, 5, 6]}
                    \item \texttt{k[:4]} returns \texttt{[1, 2, 3, 4]}
                \end{itemize}
            \item Membership operator \texttt{in} asks whether item is in list.
                \begin{itemize}
                    \item \texttt{3 in [1,2,3,4,5,6]} returns \texttt{True}
                    \item \texttt{len([1,2,3,4,5,6])} returns \texttt{6}
                \end{itemize}
        \end{itemize}
    \end{multicols}

% Lis tMethods
\subsection{List Methods}
    \begin{multicols}{2}
        \begin{itemize}
            \item \texttt{append()} adds item at end of list
            \item \texttt{insert(i, item)} inserts item at i$^{\mathrm{th}}$ pos of list
            \item \texttt{extend(iterable)} appends all items in iterable
            \item \texttt{pop()} removes \& returns last item in list
                \begin{itemize}
                    \item \texttt{pop(i)} removes \& returns i$^{\mathrm{th}}$ element in list
                \end{itemize}
            \item \texttt{del list[i]} removes i$^{\mathrm{th}}$ element in list
                \begin{itemize}
                    \item \texttt{del k[2]} deletes item at index 2 from k
                \end{itemize}
            \item \texttt{remove(item)} removes 1$^{\mathrm{st}}$ occurrence of item
            \item \texttt{sort()} modifies list to be sorted
            \item \texttt{reverse()} reverses order of items in list
            \item \texttt{count(item)} returns number of occurrences of item in list
            \item \texttt{index(item)} returns index at 1$^{\mathrm{st}}$ occurrence of item
        \end{itemize}
    \end{multicols}

\begin{lstlisting}
    >>> list('CMPUT')
    ['C', 'M', 'P', 'U', 'T']
    
    >>> '1,2,3,,5'.split(',')
    ['1', '2', '3', '', '5']
    >>>'the cat sat on the mat'.split()
    ['the', 'cat', 'sat', 'on', 'the', 'mat']
    >>>'the,cat,sat,on,the,mat'.split(',',3)
    ['the', 'cat', 'sat', 'on,the,mat']
    
    >>> ' '.join(['1','2','3','4','5'])
    '1 2 3 4 5'
    >>> ''.join(['1','2','3','4','5'])
    '12345'
    >>>  '**'.join(['1','2','3','4','5'])
    '1**2**3**4**5'
    
    >>> x = [1,2,3,4,5]
    >>> x.reverse()
    >>> x
    [5, 4, 3, 2, 1]
    
    # Note that del x[len(x)-1] removes the same value as x.pop()
    # However, del x[len(x)-1] != x.pop()
    
\end{lstlisting}

% Bio;t-In Types: Set, Tuple
\subsection{Built-In Types: \texttt{tuple, set}}
    \begin{multicols}{2}
        \begin{itemize}
            \item Tuple is immutable list
                \begin{itemize}
                    \item ex: (2, True, 'cat', [1,2,3], 3.5)
                \end{itemize}
            \item Can't change content of tuple, but can change mutable objs in tuple
                \begin{itemize}
                    \item i.e., can change content of set in tuple
                \end{itemize}
        \item Set is unordered collection of unique immutable objs, but set itself is mutable
            \begin{itemize}
                \item ex: {2, True, 'cat', 3.5}
                \item \textbf{CANNOT include lists in sets}
            \end{itemize}
        \item Sets do not support indexing
        \item Sets support methods like:
            \begin{itemize}
                \item \texttt{union} or \texttt{|}
                \item \texttt{intersection} or \texttt{\&}
                \item \texttt{issubset} or \texttt{<=}
                \item \texttt{difference} or \texttt{-}
                \item \texttt{add(item), remove(item), clear(), pop()}
            \end{itemize}
        \end{itemize}
    \end{multicols}

\begin{lstlisting}
    >>> k = (2, True, 'cat', [1,2,3])
    >>> print(k[2])
    >>> k[2].append(4)
    >>> print(k[2])
    [1, 2, 3]
    [1, 2, 3, 4]
\end{lstlisting}

% Aliasing
\section{Aliasing}
\begin{itemize}
    \item \texttt{x=y} does NOT make copy of y
    \item \texttt{x=y} makes x ref same obj that y refs CURRENTLY
    \item Use aliasing ONLY as 2nd name for MUTABLE obj.
        \begin{itemize}
            \item Aliasing for immutable objs is tricky.
        \end{itemize}
    \item Aliasing can cause problems:
\end{itemize}
\vspace{-1em}
\begin{lstlisting}
    >>> first_var = 'CMPUT'
    >>> second_var = first_var
    >>> first_var = first_var + '275'
    >>> print(first_var)
    >>> print(second_var)
    CMPUT 275
    CMPUT
\end{lstlisting}

\end{document}
