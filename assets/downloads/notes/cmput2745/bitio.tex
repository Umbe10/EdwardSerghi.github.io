\documentclass{article}
\usepackage[utf8]{inputenc}
\usepackage{multicol}
\usepackage[utf8]{inputenc}
\usepackage[T1]{fontenc}
\usepackage[margin=1in]{geometry}
\usepackage{hyperref}
\hypersetup{
    colorlinks,
    citecolor=blue,
    filecolor=blue,
    linkcolor=blue,
    urlcolor=blue
}
\usepackage{graphicx}
\usepackage{mathtools}
\usepackage{soul}
\usepackage{csvsimple}
\usepackage{hanging}
\usepackage{multicol}
\usepackage{amsmath}
\usepackage{listings}
\usepackage{amssymb}
\usepackage{float}
\usepackage{longtable}
\usepackage{pgfplotstable,filecontents}
\pgfplotsset{compat=1.9}
\usepackage[compact]{titlesec}
\titlespacing{\section}{0pt}{*0}{*0}
\titlespacing{\subsection}{0pt}{*0}{*0}
\titlespacing{\subsubsection}{0pt}{*0}{*0}
\usepackage{xcolor}
\usepackage{enumitem}
 
\definecolor{codegreen}{rgb}{0,0.6,0}
\definecolor{codegray}{rgb}{0.5,0.5,0.5}
\definecolor{codepurple}{rgb}{0.58,0,0.82}
\definecolor{backcolour}{rgb}{0.95,0.95,0.92}
 
\lstdefinestyle{mystyle}{
    backgroundcolor=\color{backcolour},   
    commentstyle=\color{codegreen},
    keywordstyle=\color{magenta},
    numberstyle=\tiny\color{codegray},
    stringstyle=\color{codepurple},
    basicstyle=\ttfamily\footnotesize,
    breakatwhitespace=false,         
    breaklines=true,                 
    captionpos=b,                    
    keepspaces=true,                 
    numbers=left,                    
    numbersep=5pt,                  
    showspaces=false,                
    showstringspaces=false,
    showtabs=false,                  
    tabsize=2
}
 
\lstset{style=mystyle}

\setlength\parindent{0pt}

\frenchspacing

\title{Python 3 Exceptions bitio.py Quick Reference}
\author{Eddie Guo}
\date{October 2019}


\begin{document}
\lstset{language=Python}
\maketitle

% Introduction to Recursion
\section{Introduction to bitio.py}
% Topics Covered
\subsection{Topics Covered}
    \begin{multicols}{2}
        \begin{enumerate}[label=(\roman*)]
            \item BitReader
            \item EOFError
            \item BitWriter
        \end{enumerate}
    \end{multicols}

% BitReader
\section{BitReader}
    \begin{multicols}{2}
        \begin{itemize}
            \item \texttt{BitReader(input\_stream)}
                \begin{itemize}
                    \item Creates instance of BitReader class
                    \item This obj will read from input\_stream
                \end{itemize}
            \item \texttt{readbit()} reads next bit in input\_stream, \& returns it as 1 or 0
            \item \texttt{readbits(n)} reads n bits \& returns them as seq of bits eval as integer
        \end{itemize}
    \end{multicols}
\vspace{-2em}
\begin{lstlisting}
import bitio

with open('simple.txt', 'rb') as fin:
    mybitreader = bitio.BitReader(fin)
    
    # read in a byte, one bit at a time
    for i in range(8):
        my_bit = mybitreader.readbit()
        print(my_bit, end = '')
    print()
    
    # read in a byte all at once
    my_bite = mybitreader.readbits(8)
    print(my_byte)
\end{lstlisting}

% How to Read to End of File?
\subsection{How to Read to End of File?}
    \begin{itemize}
        \item EOFError raised when there are no more bits to read from the input\_stream
    \end{itemize}
\vspace{-1em}
\begin{lstlisting}
import bitio

with open('simple.txt', 'rb') as fin:
    mybitreader = bitio.BitReader(fin)
    end_of_file = False
    
    while note end_of_file:
        try:
            bit = mybitreader.readbit()
            print(bit, end = '')
        except EOFError:
            end_of_file = True
\end{lstlisting}

% BitWriter
\section{BitWriter}
    \begin{multicols}{2}
        \begin{itemize}
            \item \texttt{BitWriter(output\_stream)}
                \begin{itemize}
                    \item Creates instance of BitWriter class
                    \item This obj will write to output\_stream
                \end{itemize}
            \item \texttt{writebit(bit)}
                \begin{itemize}
                    \item If bit is True, writes 1 to output\_stream
                    \item If bit is False, write 0 to output\_stream
                \end{itemize}
            \item \texttt{writebits(integer\_value, n)} writes the n least significant bits of integer\_value to output\_stream starting w/ most significant of these bits
            \item \texttt{flush()}
                \begin{itemize}
                    \item Forces any bits waiting in buffer to output\_stream
                    \item ALWAYS call when finished writing to write any partial bytes to output\_stream
                    \item Any incomplete bytes automatically padded w/ extra 0s in least significant bits
                \end{itemize}
        \end{itemize}
    \end{multicols}
\begin{lstlisting}
import bitio

seq1 = '01101000'
seq2 = [ord('E'), ord('L'), ord('L'), ord('O')]

with open('message.txt', 'wb') as fout:
    mybitwriter = bitio.BitWriter(fout)
    
    for single_bit in seq1:
        if single_bit == '1':
            mybitwriter.writebit(True)
        elif single_bit == '0':
            mybitwriter.writebit(False)
    mybitwriter.flush()  # don't forget at the end!
    
    for single_byte in seq2:
        mybitwriter.writebits(single_byte, 8)
    mybitwriter.flush() # don't forget at the end!
    
# output
hELLO
\end{lstlisting}
