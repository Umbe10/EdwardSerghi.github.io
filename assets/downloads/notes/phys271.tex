\documentclass[twocolumn]{article}
\usepackage[margin=0.75cm]{geometry}
\usepackage{hyperref}
\hypersetup{
    colorlinks,
    citecolor=blue,
    filecolor=blue,
    linkcolor=blue,
    urlcolor=blue
}

\usepackage{graphicx, multicol, wrapfig, caption, multirow, mathtools, amsfonts, booktabs, siunitx, accents, physics}
\setlength{\columnseprule}{.75pt}
\def\columnseprulecolor{\color{black}}
\newcommand{\overbar}[1]{\mkern 1.5mu\overline{\mkern-1.5mu#1\mkern-1.5mu}\mkern 1.5mu}

\setlength{\parindent}{0pt}
\setlength{\parskip}{6pt}

\everymath{\displaystyle}

\title{
	\vspace{-2em}
	\normalsize \textbf{PHYS 271 Formula Sheet} \\
	\small Eddie Guo \\
	\dotfill
	\vspace{-5em}
}
\date{}

\begin{document}
\maketitle

\small

\textbf{Constants}

$h = \SI{6.6261e-34}{J.s} = \SI{4.135e-15}{eV.s}$

$\hbar = \frac{h}{2\pi} = \SI{1.0546e-34}{J.s} = \SI{6.6582e-16}{eV.s}$

$c = \SI{3e8}{m/s}$ \hfill $\epsilon_0 = \SI{8.854e-12}{F/m}$ \hfill $\mu_0 = 4\pi\SI{e-7}{Wb/A.m}$

$\SI{1}{J} = \frac{1}{e}\ \text{eV}$ \hfill $\SI{1}{J} = \SI{2.39e-4}{kcal}$ \hfill $e = \SI{1.6022e-19}{C}$

$k = \SI{1.38e-23}{J/K}$ \hfill $R_H = \SI{1.097e7}{m^{-1}}$ \hfill $N_A = 6.022 \times 10^{23}$

$u=\SI{1.661e-27}{kg} = \SI{931.494}{MeV/c^2}$

$m_e = \SI{9.109e-31}{kg} = \SI{0.511}{MeV/c^2}$

$m_p = \SI{1.673e-27}{kg} = \SI{938.27}{MeV/c^2}$

$m_n = \SI{1.675e-27}{kg} = \SI{939.57}{MeV/c^2}$

$\mu_B = \SI{9.274e-24}{J/T} = \SI{5.788e-5}{eV/T}$

\vspace{-.5em}
\dotfill

\textbf{Special Relativity}


\textit{Galilean transforms}: $t' = t$, \hfill $x' = x + vt$, \hfill $y' = y$, \hfill $z' = z$

\textit{Inertial frame}: A reference frame in which a body at rest remains at rest and a body in motion moves at a constant speed in a straight line unless acted upon by an outside force.

\textit{Postulates}: \vspace{-1em}

\begin{enumerate}
    \item The laws of physics are the same in all inertial frames of reference.
    \item Speed of light same in all inertial frames.
\end{enumerate} \vspace{-.5em}

\textit{Relativity of simultaneity}: Two spatially-separated events simultaneous in one reference frame are not simultaneous in any other inertial frame moving relative to the first.

If events occur at same spatial location, simultaneous in all frames.

\vspace{-.5em}
\dotfill

\textit{Time dilation (proper time)}: $\Delta \tau = \frac{\Delta t}{\gamma}$ \hfill $\gamma = \frac{1}{\sqrt{1 - \beta^2}},\ \beta = \frac{v}{c}$

\vspace{-.5em}
\begin{itemize}
    \item Stationary frame experiences longer time relative to moving frame. $\Delta \tau$ is the shortest possible interval.
    \item $\Delta \tau$ is time interval btw two events is time interval measured by an observer for whom both events occur at the same location.
\end{itemize} \vspace{-.5em}

\textit{Length contraction (proper length)}: $L_0 = \gamma L$
\vspace{-.5em}
\begin{itemize}
    \item $L_0$ is length of obj measured by an observer at rest relative to it.
    \item $L_0$ is longest possible length.
\end{itemize} \vspace{-.5em}

Tip: $\Delta \tau$ and $L_0$ are measurements of $S'$ by $S$

\textit{Approximation}: If $\beta \ll 1,\ \gamma \approx 1 + \beta^2/2$ \hfill $1/\gamma \approx 1 - \beta^2/2$

\textit{Radioactive decay}: $N(t) = N_0 e^{-t/\tau}$, \hfill account for time dilation

\vspace{-.5em}
\dotfill

\textit{Lorentz transform}

$\quad S \mapsto S':$ \hfill $ct' = \gamma (ct - \beta x)$, \hfill $x' = \gamma (x - \beta c t)$, \hfill $y' = y$, \hfill $z' = z$

$\quad S' \mapsto S:$ \hfill $ct = \gamma(ct' + \beta x')$, \hfill $x = \gamma(x' + \beta c t')$, \hfill $y = y'$, \hfill $z = z'$

4-vectors: $\undertilde{x} = [ct, x, y, z]^T = [ct, \mathbf{r}]^T$
\begin{equation*}
    \undertilde{x}' = \Lambda \undertilde{x}, \quad \Lambda = \begin{bmatrix} \gamma & -\beta \gamma & 0 & 0 \\ - \beta \gamma & \gamma & 0 & 0 \\ 0 & 0 & 1 & 0 \\ 0 & 0 & 0 & 1 \end{bmatrix}
\end{equation*} \vspace{-1em}

\textbf{CAUTION:} Account for the $c$ in $ct$!


% new column
\newpage


\textit{Transformations in Minkowski space}

$x' = x \sqrt{\frac{1+\beta^2}{1-\beta^2}}$ \hfill $ct' = ct \sqrt{\frac{1+\beta^2}{1-\beta^2}}$ \hfill $\tan \alpha = \frac{v}{c}$

\textit{Intervals}: $\Delta s^2 = (c \Delta t)^2 - \Delta x^2 - \Delta y^2 - \Delta z^2$ \hfill (Lorentz invariant)

\vspace{-.5em}
\begin{itemize}
    \item Timelike: $(c \Delta t)^2 > \Delta x^2 + \Delta y^2 + \Delta z^2$ \hfill $\Delta s^2 > 0$
    \item Lightlike: $(c \Delta t)^2 = \Delta x^2 + \Delta y^2 + \Delta z^2$ \hfill $\Delta s^2 = 0$
    \item Spacelike: $(c \Delta t)^2 < \Delta x^2 + \Delta y^2 + \Delta z^2$ \hfill $\Delta s^2 < 0$
\end{itemize} \vspace{-.5em}

\textit{Causality}: If events $A$ and $B$ are linked, then they have to be timelike or lightlike separated, and all observers agree on temporal order.

\vspace{-.5em}
\begin{itemize}
    \item If events timelike, their order is fixed (inside the light cone).
    \item If events lightlike, event $A$ reaches $B$ at exact instance $B$ occurs. All observers agree that $A$ occurs before $B$.
    \item If events spacelike, no causality. Not all observers agree on order (outside of the light cone).
\end{itemize} \vspace{-.5em}

\vspace{-.5em}
\dotfill

\textit{Relativistic Doppler Effect}

$f_{\text{obs}} = f_{\text{src}} \frac{\sqrt{1-\beta^2}}{1 + \beta \cos \theta}$, \hfill light moving at $\theta$ wrt observer

Directly away: $f_{\text{obs}} = f_{\text{src}} \sqrt{\frac{1-\beta}{1+\beta}}$

Directly towards: $f_{\text{obs}} = f_{\text{src}} \sqrt{\frac{1+\beta}{1-\beta}}$

Transverse: $f_{\text{obs}} = f_{\text{src}} \sqrt{1 - \beta^2}$

\dotfill

\textbf{Moving Charges}

$\lambda_p - \lambda_n = \lambda \beta^2 \gamma$ \hfill $\lambda_p$ is +ve $q$ density, $\lambda_n$ is -ve $q$ density

Assuming $\gamma \approx 1$: $\mathbf{F} = \frac{qv^2 \lambda}{2 \pi r} \mu_0\ \hat{k}$ \hfill $\lambda_n' = \lambda / \gamma, \quad \lambda_p' = \lambda \gamma$

Highly relativistic case: $\mathbf{F}' = \gamma \mathbf{F}$ \hfill $\mathbf{F}_\perp' = \gamma \mathbf{F}_\perp$ \hfill $\mathbf{F}_\parallel' = \mathbf{F}_\parallel$

\vspace{-.5em}
\dotfill

\textbf{Relativistic Mechanics}

$u_x = \frac{u_x' + v}{1 + vu_x' / c^2}$ \hfill $u_y = \frac{u'_y}{\gamma(1 + vu_x' / c^2)}$ \hfill $u_z = \frac{u_z'}{\gamma ( 1 + vu_x' / c^2)}$

$u_x' = \frac{u_x - v}{1 - vu_x/c^2}$ \hfill $u_y' = \frac{u_y}{\gamma (1-vu_x/c^2)}$ \hfill $u_z' = \frac{u_z}{\gamma (1-vu_x/c^2)}$
\begin{itemize}
    \item $u$ is velocity of event relative to $S$.
    \item $v$ is velocity of $S'$ relative to $S$.
    \item $u'$ is velocity of event relative to $S'$.
    \item $\gamma$ is wrt to $v$.
\end{itemize} \vspace{-.5em}

4-velocity: $\undertilde{v} = [\gamma c, \gamma \dot{x}, \gamma \dot{y}, \gamma \dot{z}]^T = [\gamma c, \gamma \mathbf{u}]^T$ \hfill $\undertilde{u}' = \Lambda \undertilde{u}$

4-momentum: $\undertilde{p} = \gamma m \mathbf{u}$ \hfill Forces: $\mathbf{F} = \gamma m \frac{d \mathbf u}{dt}$

$E = \gamma mc^2 = E_k + E_0$ \hfill $E_k = (\gamma - 1)mc^2$ \hfill $E_0 = mc^2$

$\undertilde{p} = \left[ \frac{E}{c}, p_x, p_y, p_z \right]^T = \left[ \frac{E}{c}, \mathbf{p} \right]^T$ \hfill $E^2 = m^2 c^4 + p^2 c^2$

Invariant mass (Lorentz invariant): $\undertilde{p} \cdot \undertilde{p} = \frac{E^2}{c^2} - p^2 = m^2 c^2$


% new page
\cleardoublepage


For $p \gg m,\ E \approx pc$ \hfill Classical physics: $E = \frac{p^2}{2m}$

For $\mathbf{u} \perp \mathbf{B},\ quB = \gamma m \frac{u^2}{R} \implies \gamma = \frac{qBR}{mu}$ \hfill $p = q B R = \gamma m u$

Approach: Set up $E$-cons and $p$-cons $\to$ sub into $E$-$p$ relationship.

\vspace{-.5em}
\dotfill

\textbf{General Relativity}

\textit{Equivalence principle:} A homogeneous gravitational field is completely equivalent to a uniformly accelerated reference frame.

\textit{Gravitational lensing:} $\alpha = \frac{4GM}{c^2 R}$

\textit{Einstein's field eqs:} $G_{\mu \nu} + g_{\mu \nu} \Lambda = \frac{8 \pi G}{c^4} T_{\mu \nu}$

\dotfill

\textbf{Charge Quantization}

\textit{J.J. Thompson}: discovered electrons (``cathode rays'').

$\quad Q = Ne \implies W = N \left( \frac{1}{2} mu^2 \right) = \frac{Q}{e} \left( \frac{1}{2} mu^2 \right)$

$\quad mu = QBR \implies \frac{e}{m} = \frac{2W}{QB^2R^2}$

\textit{Millikan}: oil drop expt to det $e$.

$\quad$Stoke's law: $v_T = \frac{2gr^2}{9 \eta} (\rho_{\text{particle}} - \rho_{\text{medium}})$

$\quad qE = mg \implies q = \frac{mg}{E} = \frac{4 \pi r^3 \rho_{\text{oil}} g}{3E}$

\dotfill

\textbf{Energy Quantization}

Wien's displacement law: $\lambda_{\text{max}} T = \SI{2.898e-3}{m.K}$

Radiation power: $P(T) = \sigma A T^4$

Radiation intensity: $R(T) = \sigma T^4$

$\quad$Stefan's constant: $\sigma = \SI{5.6703e-8}{W/(m^2.K^4)}$

$\nabla^2 E = \frac{1}{c^2} \frac{\partial^2 E}{\partial t^2}$, cubic cavity of length $L$, $E=0$ at walls:

$\quad$Modes: $N = \frac{8 \pi L^3}{3 \lambda^3}$

$\quad$Modes/$\lambda$: $\left| \frac{dN}{d \lambda} \right| = \frac{8 \pi L^3}{\lambda^4} = \frac{8 \pi V^3}{\lambda^4}$

$\quad$Modes/$(\lambda \cdot V)$: $\rho_n(\lambda) = \frac{8 \pi}{\lambda^4}$

\dotfill

\textit{Classical Blackbody Formula via Equipartition Thm}

E density/$\lambda$: $u(\lambda) = \frac{8 \pi}{\lambda^4} \langle E \rangle = \frac{8 \pi kT}{\lambda^4}$ \hfill $k = \SI{1.38e-23}{J/K}$

Rayleigh-Jeans eq: $R(\lambda) = \frac{2 \pi k Tc}{\lambda^4}$

UV catastrophe: $\rho_E(\lambda) = \int_0^\infty u(\lambda)\ d\lambda \to \infty$ \hfill (infinite E density)

$\quad$Exptl obs: $\lambda \to 0 \implies u(\lambda) \to 0$ \hfill (Planck solves this issue)

\vspace{-.5em}
\dotfill

\textit{Planck's Blackbody Formula}

Energy: $\langle E \rangle = \frac{h \nu}{\exp \left( \frac{h \nu}{kT} \right) - 1}$

E density/$\lambda$: $u(\lambda) = \frac{8\pi}{\lambda^4} \langle E \rangle = \frac{8\pi}{\lambda^4} \frac{h\nu}{\exp \left( \frac{h \nu}{kT} \right) - 1}$ \hfill $c = h\nu$


% new column
\newpage


E density/$\nu$: $u(\nu) = \frac{dE}{d\nu} = \frac{8\pi\nu^2}{c^3} \frac{h\nu}{\exp \left( \frac{h \nu}{kT} \right)-1}$

Radiant intensity: $R(\lambda, T) = \frac{2 \pi h c^2}{ \lambda^5} \left( \frac{1}{\exp \left( \frac{hc}{\lambda kT} \right) - 1} \right)$

$\quad \lambda \text{ small} \implies R(\lambda) = \frac{2\pi kTc}{\lambda^4}$ \hfill $\lambda \text{ large} \implies R(\lambda) \to 0$

$\quad R \propto \frac{1}{\lambda^5 \left[ \exp \left( \frac{hc}{\lambda k T} \right) - 1 \right]}$

\dotfill

\textbf{Photoelectric Effect}

$E_k = eV \implies$ electron energy depends on $\nu$ of light, NOT intensity

$E_{k, \text{max}} = h \nu - \phi$ \hfill $\phi = hf_c$ (set $E_k = 0$)

X-rays: $\lambda_{\text{min}} = \frac{1.2407 \times 10^{-6}}{V}$

\dotfill

\textbf{Compton Effect}

$\undertilde{p}_{\gamma, 1} = \left[ \frac{h}{\lambda_1}, \frac{h}{\lambda_1}, 0, 0 \right]$ \hfill $\undertilde{p}_{\gamma, 2} = \left[ \frac{h}{\lambda_2}, \frac{h}{\lambda_2} \cos \theta, \frac{h}{\lambda_2} \sin \theta, 0 \right]$

$\undertilde{p}_{e, 1} = \left[ m_e c, 0, 0, 0 \right]$ \hfill $\undertilde{p}_{e, 2} = \left[ \frac{E}{c}, p_e \cos \phi, -p_e \sin \phi, 0 \right]$

$\lambda_2 - \lambda_1 = \underbrace{\frac{h}{m_e c}}_{\lambda_c} (1 - \cos \theta)$ \hfill $p_\gamma = \frac{E_\gamma}{c}$ \hfill $E_\gamma = h\nu = \frac{hc}{\lambda}$

Compton shift: $\Delta \lambda = \lambda_c(1-\cos \theta)$

\vspace{-.5em}
\dotfill

\textbf{Rutherford Scattering}

Impact parameter: $b = \frac{Q q_\alpha}{4 \pi \epsilon_0 m_\alpha v^2} \cot \frac{\theta}{2} = \frac{Qq_\alpha}{4\pi\epsilon_0 (2E_k)} \left( \frac{1+\cos\theta}{1-\cos\theta} \right)$

Area: $\sigma = \pi b^2 = \pi \left( \frac{Q q_\alpha}{8 \pi \epsilon_0 E_k} \right)^2 \left( \frac{1+\cos\theta}{1-\cos\theta} \right)$

Scattering fraction: $f = \sigma n L$ \hfill $nL:$ atoms per unit area

$\quad$Atoms/unit V: $n = \frac{1000 \rho N_A}{M}$ \hfill $\rho$ (kg/m$^3$), $M$ (u)

$A$ causing $\alpha$ to scatter at angle $\theta$: $\frac{d\theta}{d \Omega} = \left( \frac{ze^2}{8 \pi \epsilon_0 E} \right)^2 \frac{1}{\sin^4(\theta/2)}$

$N = \left( \frac{A_d nL I_0}{r_d^2} \right) \left( \frac{Ze^2}{8 \pi \epsilon_0 E} \right)^2 \frac{1}{\sin^4(\theta/2)}$ \hfill $I_0$ is beam intensity

\dotfill

\textbf{Bohr Atom}

$L = mvr = n\hbar$

Quantized radius: $r_n = \frac{4 \pi \epsilon_0 n^2 \hbar^2}{m_e Z e^2}$  \hfill $m_e$ can be diff from e$^-$ mass

Quantized energy: $E_n = -\frac{m_e Z^2e^4}{8\epsilon_0 h^2} \frac{1}{n^2}$

$\quad$Ground state E for H: $E_1 = \SI{-13.6}{eV}$ \hfill $n=Z=1$ for H

$\quad$E in $n$th orbit for H: $E_n = -\frac{E_0}{n^2} = - \frac{\SI{13.6}{eV}}{n^2}$

Rydberg formula: $\frac{1}{\lambda_{mn}} = \underbrace{\frac{m_e Z^2 e^4}{8 \epsilon_0^2 h^3 c}}_{R_H} \left( \frac{1}{m^2} - \frac{1}{n^2} \right)$ \hfill $m < n$

$\quad R_H = \SI{1.097e7}{m^{-1}}$ \hfill $m=1$ is ground state

To $m=1$: Lyman (UV), $m=2$: Balmer (vis), $m=3$: Paschen (IR)


\cleardoublepage


\textbf{Wave-Particle Duality}

Photon Momentum: $p = \frac{h}{\lambda}$ \hfill $\mathbf{p}_\gamma = \hbar \mathbf{k}$ \hfill $|\mathbf{k}| = \frac{2\pi}{\lambda}$

de Broglie wavelenth: $\lambda = \frac{h}{p}$ \hfill $E = \hbar \omega$

Non-relativistic: $\lambda = \frac{h}{\sqrt{2mE_k}}$ \hfill $E_k = \frac{3}{2} kT$

Relativistic: $\lambda = \frac{\lambda_c}{\sqrt{2(E_k / E_0) + (E_k / E_0)^2}}$ \hfill $\lambda_c = \frac{h}{mc}$

Ultra-relativistic ($E_k \gg E_0,\ E \approx pc$): $\lambda = \lambda_c \frac{E_0}{E_k} = \frac{hc}{E_k}$

\dotfill

\textbf{Wave Packets}

$\psi_{xx} = \frac{1}{c^2} \psi_{tt}$ \hfill $\psi = \psi_1 + \psi_2 = 2A e^{i \bar{k} x} e^{-i \bar{\omega} t} \cos \left( \frac{1}{2} \Delta k x - \frac{1}{2} \Delta \omega t \right)$ 

\hfill $\bar{k} = \frac{k_1 + k_2}{2},\ \bar{\omega} = \frac{\omega_1 + \omega_2}{2}$

Phase velocity: $v_p = \frac{\omega}{k}$ \hfill Group velocity: $v_g = \frac{d\omega}{dk} = v_p + k \frac{dv_p}{dk}$

\textit{Uncertainty relations:} $\Delta k \Delta x \approx 1,\ \Delta \omega \Delta t \approx 1$

\textit{Heisenberg Uncertainty Principle}

$\psi(x, t) = \int_0^\infty \int_0^\infty A(k, \omega) e^{i(kx-\omega t)}\ dk\ d\omega$ \hfill (2D Fourier transform)

$\quad \sigma_x \sigma_k = \frac{1}{2},\ \Delta x = \sigma_x,\ \Delta p = \hbar \sigma_k$ \hfill $\Delta p \Delta x \geq \frac{\hbar}{2}$ \hfill $\Delta E \Delta t \geq \frac{\hbar}{2}$

\textit{Particle in a Box of Length $L$}

$(\Delta p)^2 = \langle p^2 \rangle - \langle p \rangle^2 \implies \Delta p \geq \frac{\hbar}{L}$

$E_k = \frac{\langle p^2 \rangle}{2m},\ \langle p \rangle = 0 \implies \langle p^2 \rangle = (\Delta p)^2 \implies \langle E_k \rangle \geq \frac{(\Delta p)^2}{2m} = \frac{\hbar^2}{2mL}$

\dotfill

\textbf{Single Slit Diffraction}

$\psi(\theta) = A \cos (kr - \omega t)\ \text{sinc} \left( \frac{\pi}{\lambda} a \sin\theta \right)$ \hfill $I = I_0\ \text{sinc}^2 \left( \frac{\pi}{\lambda} a \sin\theta \right)$

At min: $\Delta L = a \sin \theta = n \lambda$ \hfill $n = \pm 1, \pm 2, ...$ (NOT 0)

$\theta \approx \frac{\Delta p_y}{p_x},\ \theta \ll 1$ \hfill $\Delta p_y \geq \frac{\hbar}{2 \Delta x},\quad \Delta x = \frac{a}{2},\quad p_x = \frac{h}{\lambda}$

\begin{figure}[h]
    \centering
    \includegraphics[width=0.5\columnwidth]{Figures/single_slit_intensity.jpeg}
\end{figure}

\vspace{-.5em}
\dotfill

\begin{wrapfigure}[7]{r}{0.35\columnwidth}
    \includegraphics[height=0.35\columnwidth]{Figures/double_slit.png}
\end{wrapfigure}

\textbf{Double Slit Diffraction}

\textit{Narrow Slit}

$\psi(\theta) = 2A \cos(\frac{\pi d}{\lambda} \sin \theta) \cos(kr - \omega t)$

$I = I_0 \cos^2 \left( \frac{\pi d}{\lambda} \sin \theta \right)$

For $m=0, \pm 1, \pm 2, ...$ \vspace{-.5em}
\begin{itemize}
    \item Max $I$: $d \sin \theta = m\lambda$
    \item Min $I$: $d \sin \theta = \left( m + \frac{1}{2} \right) \lambda$
\end{itemize}


\newpage


\textbf{Schr\"{o}dinger Equation}

$i\hbar \frac{\partial \psi}{\partial t} = -\frac{\hbar^2}{2m} \frac{\partial^2 \psi}{\partial x^2} + V(x) \psi(x, t)$ \hfill $\frac{\partial^2 \psi}{\partial t^2} = c^2 \nabla^2 \psi$

$\psi(x, t) = A e^{i(kx-\omega t)}$ \hfill $\int_{-\infty}^\infty |\psi|^2\ dx = 1,\quad |\psi|^2 = \psi^* \psi$

$\psi(x, t) = \Psi(x) \phi(t)$

$\quad \implies \phi(t) = e^{-iEt/\hbar}$

$\quad \implies -\frac{\hbar^2}{2m} \frac{d^2 \Psi}{dx^2} + V(x) \Psi(x) = E \Psi(x)$ \hfill (time-indep Schr\"o eq)

\vspace{-.5em}
\dotfill

\textit{Infinite Square Well}

$-\frac{\hbar^2}{2m} \frac{d^2 \Psi}{dx^2} = E \Psi(x) \implies \frac{d^2 \Psi}{dx^2} = - \underbrace{\frac{2mE}{\hbar^2}}_{k^2} \Psi(x)$

$\Psi(0) = \Psi(L) = 0,\quad V(x) = 0,\ 0 < x < L,\quad \int_0^L |\psi(x)|^2\ dx = 1$

$\implies \Psi(x) = \sqrt{\frac{2}{L}} \sin(kx) \implies \Psi_n(x) = \sqrt{\frac{2}{L}} \sin\left( \frac{n\pi x}{L} \right),\ n \in \mathbb{Z}$

$\implies k = \frac{n\pi}{L} \implies p = \hbar k = \frac{n\hbar \pi}{L} \implies E_n = \frac{p_n^2}{2m} = \frac{n^2 \hbar^2 \pi^2}{2mL^2}$

$\psi(x,t) = \sqrt{\frac{2}{L}} \sin \left( \frac{n\pi x}{L} \right) e^{-i\omega_n t} = \sqrt{\frac{2}{L}} \sin \left( \frac{n\pi x}{L} \right) e^{-iE_n t / \hbar}$

\dotfill

\textit{Finite Potential Well}

Inside well: $-\frac{\hbar^2}{2m} \frac{d^2 \Psi}{dx^2} = E \Psi(x)$

$\quad \Psi(x) = A_1 e^{ikx} + A_2 e^{-ikx},\quad k = \frac{\sqrt{2mE}}{\hbar}$

Outside well: $-\frac{\hbar^2}{2m} \frac{d^2 \Psi}{dx^2} = (E-V_0) \Psi(x)$

$\quad \Psi(x) = B_1 e^{\alpha x} + B_2 e^{-\alpha x},\quad \alpha = \frac{\sqrt{2m(V_0 - E)}}{\hbar}$

Well of length $x \in [-a, a]$ with potential $V_0$:

$\quad B_1 = 0$ for $x > a$ and $B_2 = 0$ for $x < -a$

$\quad$At $x=a$, $A_1 e^{ika} + A_2 e^{-ika} = B_2 e^{-\alpha a}$

$\quad$At $x=-a$, $A_1 e^{-ika} + A_2 e^{ika} = B_1 e^{-\alpha a}$

$\quad \implies \tan ka = \frac{\alpha}{k},\ -\cot ka = \frac{\alpha}{k}$

\dotfill

\textbf{Quantum Mechanics}

Expectation: $\langle x \rangle = \int_{-\infty}^\infty \psi^*(x, t) x \psi(x, t)\ dx$

Momentum operator: $\hat{p} = \frac{\hbar}{i} \frac{\partial}{\partial x}$ \hfill $\langle p \rangle = \int_{-\infty}^\infty \psi^* \left( \frac{\hbar}{i} \frac{\partial}{\partial x} \right) \psi\ dx$

Hamiltonian operator: $\hat{H} = \hat{T} + \hat{V} = \frac{\hat{p}^2}{2m} + \hat{V}$

$\quad \implies \hat{H} = -\frac{\hbar^2}{2m} \frac{\partial^2}{\partial x^2} + V(x) \implies \hat{H} \Psi(x) = E \Psi(x)$

\dotfill

\textit{Quantum Simple Harmonic Oscillator}

$V(x) = \frac{1}{2} kx^2 = \frac{1}{2} \underbrace{m \omega^2}_{k} x^2$ \hfill $\hat{a}_\pm = \frac{1}{\sqrt{2\hbar m \omega}} (\mp i\hat{p} + m\omega x)$

$[A, B] = AB - BA$ \hfill $[x, \hat{p}] = i \hbar$ \hfill $[\hat{a}_{-}, \hat{a}_+] = 1$


% new page
\cleardoublepage

$[A, B] = 0 \implies$ $A$ and $B$ can be measured simultaneously.

$\hat{H} = \hbar \omega \left(\hat{a}_+ \hat{a}_{-} + \frac{1}{2} \right) = \hbar \omega \left(\hat{a}_{-} \hat{a}_{+} - \frac{1}{2} \right)$

Raising operator: $\hat{H} (\hat{a}_+ \Psi) = (E + \hbar \omega) (\hat{a} + \Psi)$

Lowering operator: $\hat{H} (\hat{a}_{-} \Psi) = (E - \hbar \omega) (\hat{a}_{-} \Psi)$

Ground state: $\hat{a}_{-} \Psi_0(x) = 0$

$E_n = \left(n + \frac{1}{2} \right) \hbar \omega$

$\Psi_n(x) = N_n e^{-\beta^2x^2/2} H_n(\beta x)$ \hfill $\beta = \sqrt{\frac{m\omega}{\hbar}}$

Physicist's Hermite polynomial: $H_n(x) = (-1)^n e^{x^2} \frac{d^n}{dx^n} e^{-x^2}$

$\quad H_0(x) = 1,\ H_1(x) = 2x,\ H_2(x) = 4x^2 - 2,\ H_3(x) = 8x^3-12x$

\vspace{-.5em}
\dotfill

\textit{Reflection \& Transmission}

Step Potential of $V_0 < E$

$x < 0: \frac{d^2 \Psi}{dx^2} = -k_1^2 \psi(x), \quad k_1 = \frac{\sqrt{2mE}}{\hbar}$

$x > 0: \frac{d^2 \Psi}{dx^2} = -k_2^2 \psi(x), \quad k_2 = \frac{\sqrt{2m(E-V_0)}}{\hbar}$

For $x < 0: \Psi_1(x) = \underbrace{A e^{ik_1 x}}_{L \to R} + \underbrace{Be^{-ik_1 x}}_{R \to L}$

For $x > 0: \Psi_2(x) = Ce^{ik_2 x}$

BC: $A + B = C, \quad k_1 A - k_1 B = k_2 C$

$R = \frac{|B|^2}{|A|^2} = \frac{(k_1 - k_2)^2}{(k_1 + k_2)^2}$ \hfill $T = \frac{k_2}{k_1} \frac{|C|^2}{|A|^2} = \frac{4k_1 k_2}{(k_1 + k_2)^2}$ \hfill $R + T = 1$

\dotfill

\textit{Quantum Tunneling}

Barrier potential of $V_0 > E$ from $x \in [0, L]$.

$x < 0: \Psi_1(x) = A e^{ik_1 x} + B e^{-i k_1 x}$

$0 < x < a: \Psi_2(x) = Ce^{-\alpha x} + De^{\alpha x}$

$x > a: \Psi_3(x) =  Fe^{ik_1 x} + Ge^{-ik_1 x}$

$T(L, E) = \frac{1}{\cosh^2(\beta L) + (\gamma/2)^2 \sinh^2(\beta L)} \approx 16\frac{E}{V_0} \left(1 - \frac{E}{V_0} \right) e^{-2\beta L}$

$\quad \gamma = \frac{\beta}{k} - \frac{k}{\beta}, \quad k = \frac{\sqrt{2mE}}{\hbar}$ \hfill $\beta = \frac{\sqrt{2m (V_0 - E)}}{\hbar}$

\dotfill

\textit{Young's Double Slit Experiment}

$\frac{d^2 \Psi}{dx^2} = -k^2 \Psi(x) \implies \Psi(x) = Ae^{ikx}$

At screen: $\Psi(x) = A(e^{ikx} + e^{ik(x+\Delta x)}) = 2A e^{ikx} e^{ik\Delta x / 2} \cos \left( \frac{k\Delta x}{2} \right)$

$|\Psi(x)|^2 = 4|A|^2 \cos^2 \left( \frac{k\Delta x}{2} \right)$

\dotfill

\textit{Polarized Light}

Consider $\rightarrow$ filter, \SI{45}{\degree} filter, and $\uparrow$ filter, where $\rightarrow$ $\perp$ $\uparrow$ filter.

$|\Psi_0|^2 = 0.5 \bra{\rightarrow} \ket{\rightarrow} + 0.5 \bra{\uparrow} \ket{\uparrow}$

$|\Psi_1|^2 = 0.5 \bra{\rightarrow} \ket{\rightarrow} \implies \Psi_1 = \frac{1}{\sqrt{2}} \ket{\rightarrow}$

% new column
\newpage

We're at \SI{45}{\degree} filter (think $\cos \theta \ket{\rightarrow} + \sin \theta \ket{\uparrow}$):

$\ket{\SI{45}{\degree}} = \frac{1}{\sqrt{2}} (\ket{\rightarrow} + \ket{\uparrow})$ \hfill $\ket{\SI{-45}{\degree}} = \frac{1}{\sqrt{2}} (\ket{\rightarrow} - \ket{\uparrow})$

$\implies \ket{\rightarrow} = \frac{1}{\sqrt{2}} (\ket{\SI{45}{\degree}} + \ket{\SI{-45}{\degree}})$ \hfill $\ket{\uparrow} = \frac{1}{\sqrt{2}} (\ket{\SI{45}{\degree}} - \ket{\SI{-45}{\degree}})$

$\implies \Psi_1 = \frac{1}{\sqrt{2}} \ket{\rightarrow} = \frac{1}{2} (\ket{\SI{45}{\degree}} + \ket{\SI{-45}{\degree}}) \implies \Psi_2 = \frac{1}{2} \ket{\SI{45}{\degree}}$

$\Psi_2 = \frac{1}{2\sqrt{2}} (\ket{\rightarrow} + \ket{\uparrow})$

$\Psi_3 = \frac{1}{2\sqrt{2}} \ket{\uparrow} \implies |\Psi_3|^2 = \frac{1}{8} \bra{\uparrow} \ket{\uparrow} = \frac{|\Psi_0|^2}{8}$

\dotfill

\textbf{Atomic Physics}

3D Sch\"odinger eq: $-\frac{\hbar^2}{2m} \nabla^2 \Psi(\mathbf{r}, t) + V(\mathbf{r}, t) \Psi(\mathbf{r}, t) = i\hbar \frac{\partial \Psi(\mathbf{r}, t)}{\partial t}$

\textit{Infinite Square Well}

$\Phi(\mathbf{r}) = \sqrt{\frac{8}{L_x L_y L_z}} \sin\left(\frac{n_x\pi x}{L_x}\right) \sin\left(\frac{n_y\pi y}{L_y}\right) \sin\left(\frac{n_y\pi y}{L_y}\right)$

$E_{n_x n_y n_z} = \frac{h^2}{8m} \left( \frac{n_x^2}{L_x^2} + \frac{n_y^2}{L_y^2} + \frac{n_z^2}{L_z^2} \right)$ \hfill $\phi(t) = e^{-iEt/\hbar}$

Ground state when $n_x = n_y = n_z = 1$. If 2 or more dimensions of box equal, then the excited states are degenerate.

If $(L_x > L_y) \land (L_x > L_Z)$, first excited state is $(n_x, n_y, n_z) = (2, 1, 1)$.

\vspace{-.5em}
\dotfill

\textit{Schr\"odinger Equation in Spherical Coordinates for Hydrogen Atom}

Separation of variables: $\Psi(r, \theta, \phi) = R(r) Y(\theta, \phi) = R(r) \Theta(\theta) \Phi(\phi)$

$Y = Y(\theta, \phi)$ is spherically harmonic
\vspace{-2em}

\begin{equation*}
    \begin{split}
        \frac{1}{R} \frac{d}{dr} \left( r^2 \frac{dR}{dr} \right) &+ \frac{2mr^2}{\hbar^2} [E - V(r)] \\
        &= -\frac{1}{\sin \theta} \left[ \frac{\sin \theta}{\Theta} \frac{d}{d\theta} \left( \sin \theta \frac{d\Theta}{d\theta} + \frac{1}{\Phi} \frac{d^2\Phi}{d\phi^2} \right) \right] = \underbrace{l(l+1)}_{\text{const}}
    \end{split}
\end{equation*} \vspace{-1.5em}

\vspace{-.5em}
\dotfill

$-\frac{\sin \theta}{\Theta} \frac{d}{d\theta} \left( \sin \theta \frac{d\Theta}{d\theta} \right) - l(l+1) \sin^2 \theta = \frac{1}{\Phi} \frac{d^2 \Phi}{d\phi^2} = \underbrace{-m^2}_{\text{const}}$

$\quad \Phi(\phi) = e^{im\phi}, \quad m \in \mathbb{Z}$

$\quad \Theta(\theta) = N_{lm} P_m^l(\cos \theta), \quad P_m^l(x) = (1-x^2)^{m/2} \frac{d^l P_m(x)}{dx^l}$

$\quad \implies l \in \mathbb{Z},\ |m| \leq l$

\vspace{-.5em}
\dotfill

$\hat{E}_k \Psi + V \Psi = E\Psi \implies \left( \frac{\hat{p}_r^2}{2m} + \frac{\hat{L}^2}{2mr^2} \right) \Psi + V\Psi = E\Psi$

$\quad \hat{p}_r^2 = -\hbar^2 \frac{1}{r^2} \frac{\partial}{\partial r} \left(r\frac{\partial}{\partial r}\right)$

$\quad \hat{L}^2 = -\hbar^2 \frac{1}{\sin^2\theta} \left[ \sin \theta \frac{\partial}{\partial \theta} \left(\sin \theta \frac{\partial}{\partial \theta}\right) + \frac{\partial^2}{\partial \phi^2} \right]$

$\implies \hat{L}^2 Y_{lm}(\theta, \phi) = l(l+1)\hbar^2 Y_{lm}(\theta, \phi)$

$\implies \hat{L}_z^2 Y(\theta, \phi) = m^2 \hbar^2 Y(\theta, \phi)$

$|\mathbf{L}| = \sqrt{l(l+1)} \hbar,\ l \in \mathbb{N}$ \hfill ($L$ quantized)

$L_z = m\hbar = \sqrt{l(l+1)} \hbar \cos \theta$ \hfill $m = 0, \pm 1, \dots, \pm l$

$\theta = \cos^{-1} \left( \frac{m}{\sqrt{l(l+1)}} \right)$ \hfill $\theta_{\min}$ if $m = l$; $2l+1$ values for $L_z$

\cleardoublepage

$-\frac{\hbar^2}{2mr^2} \frac{d}{dr} \left(r^2 \frac{dR(r)}{dr}\right) + \left[ -\frac{e^2}{4 \pi \epsilon_0 r} + \frac{\hbar^2 l(l+1)}{2mr^2} \right] R(r) = ER(r)$

$\implies E_n = -\frac{m_e}{2} \left( \frac{e^2}{4\pi\epsilon_0\hbar} \right)^2 \frac{1}{n^2} = -\frac{E_0}{n^2}$ \hfill $E_0 = \SI{13.6}{eV}$

\dotfill

\textit{Quantum Numbers}

\vspace{-.5em}
\begin{itemize}
    \item $n$: principal quantum number (QN); dets E lvl of $e^-$; $n \in \mathbb{N} \setminus \{0\}$
    \item $l$: angular momentum QN; gives $|\mathbf{L}|$; $l = 0, 1, \dots, (n-1)$
    \subitem--- $l=0:\ s,$ \hfill $l=1:\ p,$ \hfill $l=2:\ d,$ \hfill $l=3:\ f$
    \item $m$: angular momentum projection QN; $m = 0, \pm 1, \dots, \pm l$
    \subitem--- aka magnetic QN ($m_l$)
    \item $s$: spin QN; 1/2 for fermions
    \item $m_s$: spin projection QN; $m_s = \pm 1/2$
\end{itemize} \vspace{-.5em}

For $l=0$: $P(r)\ dr = |\psi_{n00}|^2 4\pi r^2\ dr$

\vspace{-.5em}
\dotfill

\textit{Orbital Angular Momentum}

$\mu = IA = \frac{1}{2} qvr$ \hfill $I = \frac{q}{T} = \frac{qv}{2\pi r}$ \hfill $A = \pi r^2$

$\mu = \frac{e}{2m_e} L = \mu_B \sqrt{l(l+1)}$ \hfill $\mu_B = \frac{e\hbar}{2m_e} = \SI{9.27e-24}{J/T}$

$\boldsymbol \mu = -\frac{e}{2m_e} \mathbf{L}$

\dotfill

\textit{Stern-Gerlach Expt \& Spin-Orbit Coupling}

$\boldsymbol \mu = -\frac{g_L \mu_B}{\hbar} \mathbf L = \frac{e}{m_e} \mathbf S$ \hfill $\mu_s = \frac{e}{2m_e} S$

$\mathbf F = -\nabla U = -\nabla(-\mathbf u \cdot \mathbf B) \implies F_z = \mu_z \frac{dB}{dz} = -\mu_B m_l g_L \frac{dB}{dz}$

\vspace{-.5em} \begin{itemize}
    \item Expected 1 line on screen but saw 2 lines.
    \item 1 line = uniform, constant field.
    \item 2 lines = non-uniform field.
\end{itemize} \vspace{-.5em}

% \vspace{-.5em}
% \dotfill

\textit{Spin angular momentum}: $|\mathbf{S}| = \sqrt{s(s+1)} \hbar$ \hfill $\mu_z = -g \mu_B m_s$

$\quad S_z = m_s \hbar,\ m_s = \pm\frac{1}{2}$ for e$^-$ \hfill $\mu_z = -\frac{e}{2m_e} = \pm \mu_B \hbar$

\vspace{-.5em}
\dotfill

\textit{Zeeman Effect}

$U = -\mu_z B = -(\underbrace{-\mu_B m_l}_{\mu_z})B = m_l \mu_B B$ \hfill $\mu_z = \mu \cos \theta$

$\quad$Splitting of energy lvls by $B_{\text{ext}}$. Larger $B$ means larger splitting.

$\text{\# of splits} = 2l+1$ \hfill E.g., $l=1 \to l=0$ has 3 lines.

Pauli exclusion principle: no two e$^-$ can be in exactly the same physical state at the exact same time.

% $N_{e^-} |_n = \sum_{l=0}^{n-1} (2l+1)(2)$

\vspace{-.5em}
\dotfill

\textit{Total Angular Momentum}

$\mathbf J = \mathbf L + \mathbf S$ \hfill $j = l+s$ or $j = l + m_s$

$J = \sqrt{j(j+1)} \hbar$ \hfill $J_z = m_j \hbar$ \hfill \# states of $m = 2j+1$

\vspace{-.5em}
\dotfill

\textbf{Nuclear Physics}

\textit{Hyperfine Splitting}

$\mathbf{F} = \mathbf{I} + \mathbf{J}$ \hfill $\mathbf{I}:$ nuclear ang mom,$\quad$ $\mathbf{J}:$ e$^-$ ang mom

$F = \sqrt{f(f+1)} \hbar,\quad f = |i-j|, \dots, i+j$

$\Delta E = g_M m_l \mu_N B_e$ \hfill $\mu_n = \frac{e\hbar}{2m_p}, \quad$nuclear magneton


\newpage


$\alpha:\ ^A_Z X \to^{A-4}_{Z-2} Y + ^4_2\alpha$ (due to quantum tunelling)

$\beta$ decay due to weak force

$\quad e^{-}$ emission: $^A_Z X \to ^A_{Z+1} Y + ^0_{-1} e^- + \overline{\nu}_e$

$\quad p^+$ emission: $^A_Z X \to ^A_{Z-1} Y + ^0_{1} e^+ + \nu_e$

$\quad e^-$ capture: $^A_Z X + ^0_{-1} e^- \to ^A_{Z-1} Y + \nu_e$

$\gamma:\ ^A_Z X^* \to ^A_Z X + \gamma$

Atomic radius: $R = R_0 A^{1/3}, \quad R_0 = \SI{1.2}{fm}, \quad A \equiv$ \# of nucleons

\vspace{-.5em}
\dotfill

\textit{Binding Energy and Radioactivity}

$E_b = \underbrace{(Z m_p + N m_n - M_A)}_{\Delta m} c^2$, \hfill $BEN = \frac{E_b}{A}$ ,\hfill $A$ is atomic \#

Fe-56 is most stable atomic nucleus, \hfill stable $N = \{ 20, 28, 50, 82, 126 \}$

$A < 56$ releases $E$ when extra nucleon added, \textit{fusion} more likely

$A > 56$ requires $E$ when extra nucleon added, \textit{fission} more likely

$\frac{dN}{dt} = -\lambda N \implies N = N_0 e^{-\lambda t},\ N = N_0 (2)^{-t/t_{1/2}}$ \hfill $t_{1/2} = \frac{\ln 2}{\lambda}$

For decay series where $b \to a$: $\frac{dN_a}{dt} = -\lambda_a N_a + \lambda_b N_b$

\vspace{-.5em}
\dotfill

\textit{Strong Force}

At short range, Coulomb potential dominates: $U_c \propto 1/r$

At $O(\SI{1e-15}{m})$, strong force dominates: $U_s \propto -1/r^m,\ m > 1$

$\quad$Force transmitted via $\pi$ meson

At short range, repulsive $U$ dominates: $U_{\text{exclusion}} \propto 1/r^n,\ n > m$

$\quad$As radius $\downarrow$, $E$ lvls $\uparrow$ spacing (think Pauli exclusion)

\vspace{-.5em}
\dotfill

\textbf{Quantum Mechanics \& Relativity}

\textit{Klein-Gordon Equation}

$\frac{\partial^2 \phi}{\partial t^2} - \nabla^2 \phi + m^2 \phi = 0$

4-vector gradient: $\partial_\mu = \left( \frac{\partial}{\partial t}, \frac{\partial}{\partial x}, \frac{\partial}{\partial y}, \frac{\partial}{\partial z} \right), \quad \partial^\mu = \partial_\mu^T$

$(\partial_\mu \partial^\mu + m^2) \phi(\mathbf x, t) = 0 \equiv (\square^2 + m^2) \phi(\mathbf x, t)$ \hfill $\square^2$ is d'Alembertian

Plane wave soln: $\phi(\mathbf x, t) = Ce^{-iEt+i\mathbf p \mathbf x} \implies (-E + p^2 + m^2) \phi = 0 \implies E = \pm \sqrt{p^2+m^2}$ \hfill issue of negative $E$

\vspace{-.5em}
\dotfill

\textit{Dirac Equation}

$i \frac{\partial \psi}{\partial t} = (-i \boldsymbol \alpha \cdot \nabla \beta m) \psi(\mathbf x, t)$ or $E \psi (\boldsymbol \alpha \cdot \mathbf p + \beta m) \psi$

$\implies -\frac{\partial^2 \psi}{\partial t^2} = (-\nabla^2 + m^2) \psi(x)$ or $E^2 \psi = (\mathbf p^2 + m^2) \psi$

\begin{itemize}
    \item $\psi$ is now a 4-component vector called a spinor.
    \item Solutons to Dirac eq have positive definite probability density.
    \item Dirac: spin-1/2 particles; Klein-Gordon: spin-0 particles.
    \item Dirac says $E$ lvls symmetric about $E=0$. To prevent +ve $E$ e$^-$ going to -ve $E$ states, all -ve $E$ states are filled (Pauli exclusion) $\implies$ vaccum is sea of $E < 0$ e$^-$.
    \item Anti-particles due to absence of e$^-$ w/ $E < 0$ (i.e., a hole).
\end{itemize} \vspace{-.5em}

Feynman interpretation: -ve $E$ particles propagating back in time $\equiv$ +ve $E$ particles propagating forward in time.


\end{document}
